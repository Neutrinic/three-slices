\documentclass[12pt,a4paper]{article}

\usepackage{amsmath,amssymb,amsthm,mathrsfs}
\usepackage[margin=1in]{geometry}
\usepackage[colorlinks=true,linkcolor=blue,citecolor=blue,urlcolor=blue]{hyperref}
\usepackage{enumitem}
\usepackage{tikz-cd}
\usepackage{booktabs}

% Theorem environments
\newtheorem{theorem}{Theorem}[section]
\newtheorem{proposition}[theorem]{Proposition}
\newtheorem{lemma}[theorem]{Lemma}
\newtheorem{corollary}[theorem]{Corollary}
\newtheorem{conjecture}[theorem]{Conjecture}
\theoremstyle{definition}
\newtheorem{definition}[theorem]{Definition}
\newtheorem{example}[theorem]{Example}
\newtheorem{remark}[theorem]{Remark}

% Notation shortcuts
\newcommand{\C}{\mathbb{C}}
\newcommand{\R}{\mathbb{R}}
\newcommand{\Z}{\mathbb{Z}}
\newcommand{\N}{\mathbb{N}}
\newcommand{\HH}{\mathcal{H}}
\newcommand{\DD}{\mathcal{D}}
\newcommand{\FF}{\mathcal{F}}
\newcommand{\VV}{\mathcal{V}}
\newcommand{\g}{\mathfrak{g}}
\newcommand{\fk}{\mathfrak{k}}
\newcommand{\fp}{\mathfrak{p}}
\newcommand{\fa}{\mathfrak{a}}
\newcommand{\fn}{\mathfrak{n}}
\newcommand{\SO}{\mathrm{SO}}
\newcommand{\SL}{\mathrm{SL}}
\newcommand{\SU}{\mathrm{SU}}
\newcommand{\Sp}{\mathrm{Sp}}
\newcommand{\GL}{\mathrm{GL}}
\newcommand{\Ad}{\mathrm{Ad}}
\newcommand{\ad}{\mathrm{ad}}
\newcommand{\Tr}{\mathrm{Tr}}
\newcommand{\Vol}{\mathrm{Vol}}
\newcommand{\rank}{\mathrm{rank}}
\newcommand{\Hom}{\mathrm{Hom}}
\newcommand{\End}{\mathrm{End}}
\newcommand{\Res}{\mathrm{Res}}

\title{The Cauchy--Szeg\H{o} Kernel as Holomorphic Filter:\\
  Simultaneous Reflection Positivity via Analytic Continuation\\
  Through Type~IV Bounded Symmetric Domains}

\author{A.~Abrahams}

\date{February 2026}

\begin{document}

\maketitle

\begin{abstract}
We prove a simultaneous reflection-positivity transfer between 
the Euclidean, Lorentzian, and split real forms of complexified 
$d$-dimensional spacetime by analytic continuation through the 
Type~IV bounded symmetric domain 
$D^N_{IV} \cong \SO_0(2,N)/S(O(2) \times O(N))$, where 
$N = 2d - 2$. The Cauchy--Szeg\H{o} kernel 
$S(z,\zeta) = c_N [1 - 2\langle z, \bar{\zeta}\rangle 
+ (z \cdot z)\overline{(\zeta \cdot \zeta)}]^{-N/2}$ acts as a 
Hardy-space projection (``holomorphic filter'') selecting 
precisely the positive-energy boundary data that extend 
holomorphically to the tube domain. We show the scalar 
Hardy-space $K$-spectrum on the Lie sphere is uniformly 
$\delta$-even---a rank-2 lattice parity constraint 
($k + |\lambda| = 2m_1$, always even)---so the $\delta$-odd 
obstruction identified in the algebraic approach of a companion 
paper cannot arise from boundary $K$-types. In the vector-valued 
setting the remaining sign is entirely fiber-theoretic, controlled 
by the $\delta$-intertwiner $J_\tau$; for 
$J_\tau = +\mathrm{id}$ (all bosonic/tensor fields and Dirac 
fermions in even dimensions) the Szeg\H{o} transfer preserves 
reflection positivity across all real forms. This establishes the $K$-type label compatibility required for simultaneous reflection positivity. The operator-level
reconstruction additionally requires the antiunitary/standard
subspace framework of Neeb--\'Olafsson, with the covering group
on which the BGL net is local determined by $d \bmod 4$.
\end{abstract}

\tableofcontents

%=============================================================================
\section{Introduction}
\label{sec:intro}
%=============================================================================

\subsection{The Problem}

The Osterwalder--Schrader reconstruction theorem \cite{OS1973,OS1975} establishes 
that a Euclidean quantum field theory satisfying reflection positivity determines a 
unique relativistic (Lorentzian) quantum field theory. The converse passage, from 
Lorentzian to Euclidean, is mediated by Wick rotation $t \mapsto -i\tau$, which 
is an analytic continuation in the time coordinate.

In \cite{Bridge}, we generalised this correspondence to the setting of complexified 
spacetime $\C^d$, where the Euclidean and Lorentzian theories are supplemented by 
a third formulation on the split-signature real form $\R^{d/2, d/2}$. The three 
real forms---Euclidean $\R^d$, Lorentzian $\R^{1,d-1}$, and split 
$\R^{d/2,d/2}$---are the fixed-point sets of three involutions $\alpha$, $\beta$, 
$\gamma$ on $\C^d$ whose composition relations form a Klein four-group $V_4$.

The \emph{algebraic transfer} developed in \cite{Bridge} maps correlation functions 
between these formulations via the $V_4$ structure. For $d = 4$, this transfer is 
\emph{exact}: every state in the Hilbert space of one formulation maps to a state 
in each of the others, preserving the inner product and hence reflection positivity. 
The key reason is that the relevant symmetry group $\SO(4,\C)$ decomposes as 
$\SL(2,\C) \times \SL(2,\C)$, and both factors carry the same representation 
content under $\delta$---the parity involution distinguishing the two $\SL(2,\C)$ 
factors.

For $d > 4$, the algebraic transfer encounters a precise obstruction. The group 
$\SO(2d-2, \C)$ no longer decomposes as a product of isomorphic simple factors. 
The algebraic transfer map, which operates on the full Hilbert space including 
fiber representations, produces sign reversals in the inner product for certain 
states. In \cite{Bridge}, these were identified as ``$\delta$-odd'' states 
(where $\delta = \alpha \circ \theta$), and the sign reversal was shown to 
destroy the reflection positivity required for a physical Hilbert space 
interpretation.

This left open the central question formulated as Conjecture~8.1 of \cite{Bridge}:

\begin{conjecture}[Conjecture 8.1 of \cite{Bridge}]
\label{conj:main}
Let $\{\mathcal{W}_n^{(\sigma)}\}$ be the Wightman distributions of a 
reflection-positive QFT on the real form $\R^{p,q}$ of $\C^d$ (with 
$p + q = d$, $\sigma$ labelling the signature). Then there exist Wightman 
distributions $\{\mathcal{W}_n^{(\sigma')}\}$ for every other real form 
$\R^{p',q'}$ such that:
\begin{enumerate}[label=(\roman*)]
  \item $\mathcal{W}_n^{(\sigma')}$ satisfies the Osterwalder--Schrader 
    axioms with respect to the reflection positivity involution natural 
    to that signature;
  \item $\mathcal{W}_n^{(\sigma')}$ is obtained from $\mathcal{W}_n^{(\sigma)}$ 
    by an explicit integral transform;
  \item The transform preserves the spectrum condition and cluster properties.
\end{enumerate}
\end{conjecture}

\subsection{The Strategy: Analysis Over Algebra}

The failure of the algebraic approach does not mean 
the conjecture is false---it means a purely algebraic proof strategy is 
insufficient. The present paper resolves the conjecture affirmatively 
(for fields with positive $\delta$-structure; see 
Definition~\ref{def:delta-structure}) by 
replacing the algebraic transfer with an \emph{analytic} one: continuation 
through the interior of a bounded symmetric domain.

The key insight is that the three real forms of $\C^d$ are boundary components 
of a single complex domain---the Type~IV bounded symmetric domain 
$D^N_{IV}$ (the Lie ball), where $N = 2d - 2$. This domain, isomorphic 
to the quotient $\SO_0(2, N)/S(O(2) \times O(N))$, has the remarkable 
property that its Shilov boundary is the Lie sphere 
$(S^1 \times S^{N-1})/\Z_2$, which carries the Lorentzian data, 
while the Riemannian symmetric space $\SO_0(2,N)/S(O(2) \times O(N))$ 
embedded in the interior (via restriction to the imaginary axis in the 
tube realisation) carries the Euclidean data.

\begin{proposition}[Spacetime--domain correspondence]
\label{prop:spacetime-domain}
For $d$-dimensional spacetime, the relevant Type~IV domain has 
parameter $N = 2d - 2$. The symmetry group is $G = \SO_0(2, 2d-2)$, 
with maximal compact subgroup $K = S(O(2) \times O(2d-2))$. The 
tube domain $T_\Omega = \R^N + i\Omega \subset \C^N$ has real 
dimension $2N$. The three real forms of $\C^d$ embed as 
distinguished totally real submanifolds of dimension $d$ inside 
$T_\Omega$ (or its boundary):
\begin{itemize}
  \item \emph{Lorentzian} $\R^{1,d-1}$: embedded via the 
    $V_4$-framework of \cite{SplitWedge} as a distinguished 
    $d$-dimensional totally real submanifold of the Shilov 
    boundary $\check{S} \cong (S^1 \times S^{N-1})/\Z_2$ 
    (which itself has dimension $N \geq d$);
  \item \emph{Euclidean} $\R^d$: the imaginary axis $i\Omega$ 
    (the ``Euclidean section'' at $x = 0$), a $d$-dimensional 
    totally real submanifold of the tube interior;
  \item \emph{Split} $\R^{d/2,d/2}$: the totally real 
    submanifold obtained by restricting to the $\beta$-fixed 
    locus in $\C^d$, which embeds into $T_\Omega$ via the 
    analytic continuation $x \mapsto x + i\epsilon$ with 
    $\epsilon$ in the split-signature causal cone 
    $\Omega_{\mathrm{split}} \subset \Omega$.
\end{itemize}
In each case, the Wightman distributions of a QFT on the 
$d$-dimensional real form are \emph{pulled back} to distributions 
on $\R^N$ coordinates via the embedding, and the boundary 
values of holomorphic functions in $T_\Omega$ along each 
submanifold are well-defined as tempered distributions.
\end{proposition}

\begin{proof}[Proof of Proposition~\ref{prop:spacetime-domain}]
The Lorentzian and Euclidean embeddings are standard (see \cite{FaKo1994, Hua1963}). 
We verify the split embedding.

In the Jordan-algebraic spectral decomposition of the rank-2 
algebra $V_N$, any element $y \in V_N$ decomposes as 
$y = \lambda_1 e_1 + \lambda_2 e_2$ with spectral values 
$\lambda_1, \lambda_2 \in \R$ and orthogonal idempotents 
$e_1, e_2$. The Lorentz cone is 
$\Omega = \{y : \lambda_1 > 0,\, \lambda_2 > 0\}$, 
equivalently $\{y : y_0 > \|y'\|\}$ in the 
$(y_0, y') \in \R \times \R^{N-1}$ splitting.

The split-signature real form $\R^{d/2,d/2}$ embeds into 
$\C^N$ via the restriction of the complexification map 
to the $\beta$-fixed locus. Concretely, this is the 
$\SO(4,\C)$ rotation $\Lambda$ of \cite[Lemma~3.4]{SplitWedge} 
that maps the Lorentzian forward cone $V^+_L$ into the split 
forward cone $V^+_S$. In the tube domain language, this 
rotation maps the Lorentzian tube $T_L = \R^d + iV^+_L$ 
into the split tube $T_S = \R^d + iV^+_S$. By 
\cite[Lemma~3.4]{SplitWedge}, $T_S \subset T'$ (the permuted 
extended tube), and therefore $T_S$ embeds into $T_\Omega$ 
via the spacetime-to-domain correspondence.

The inclusion $\Omega_{\mathrm{split}} \subset \Omega$ at the 
level of spectral values is immediate: the split forward cone 
$\Omega_{\mathrm{split}}$ consists of elements $\epsilon$ with 
both spectral values positive (corresponding to positivity in 
both timelike directions of $\R^{d/2,d/2}$), which is precisely 
the defining condition for $\Omega$. The totally real condition 
holds because $\R^{d/2,d/2}$ is the fixed-point set of the 
antiholomorphic involution $\beta: z \mapsto \bar{z}^\beta$ 
(the split conjugation), and the fixed-point set of an 
antiholomorphic involution on a complex manifold is always 
totally real.
\end{proof}

The Cauchy--Szeg\H{o} kernel $S(z, \zeta)$ of this domain is the 
reproducing kernel for the Hardy space $H^2(D^N_{IV})$. It provides 
an explicit integral formula that reconstructs a holomorphic function 
in the interior from its boundary values on the Shilov boundary:
\begin{equation}
\label{eq:szego-reconstruction}
  f(z) = \int_{\check{S}} S(z, \zeta) f(\zeta)\, d\sigma(\zeta), 
  \qquad z \in D^N_{IV},\; f \in H^2(\check{S}).
\end{equation}
We call the Szeg\H{o} projection a \emph{holomorphic filter} because 
it selects exactly the positive-energy, holomorphically extendable 
components of boundary data and discards all modes in the orthogonal 
complement $L^2(\check{S}) \ominus H^2(\check{S})$. This filter is 
not an additional axiom: it is the Hardy-space boundary-value 
condition already implicit in the Wightman axioms and the 
Bargmann--Hall--Wightman analytic continuation theorem. 
Concretely: the spectrum condition (positive energy) means the 
Fourier support of Wightman distributions lies in the forward 
cone $\overline{\Omega}$; by the Paley--Wiener theorem for tube 
domains \cite{SteinWeiss1971}, this is equivalent to the boundary 
values belonging to the Hardy space $H^2(\check{S})$ rather than 
the full $L^2(\check{S})$.

A key finding of the present analysis---not anticipated in 
\cite{Bridge}---is that the relationship between the holomorphic 
filter and the $\delta$-odd obstruction is more subtle than 
initially expected. In the \emph{scalar} sector, we prove that 
$\delta$-odd $K$-types do not exist on the Lie sphere at all: the 
rank-2 lattice structure forces $k + |\lambda| = 2m_1$ to be even 
for every $K$-type (Theorem~\ref{thm:disjointness}). The 
$\delta$-odd obstruction from \cite{Bridge} therefore lives 
entirely in the \emph{fiber representation} of vector-valued fields. 
For fields whose fiber carries positive $\delta$-structure 
($J_\tau = +\mathrm{id}$; this includes all bosonic/tensor fields 
and Dirac fermions in even dimensions), the combined 
$\delta$-eigenvalue on the Hardy space is uniformly $+1$, and the 
Szeg\H{o} transfer preserves reflection positivity.

Consequently, the Cauchy--Szeg\H{o} integral provides a transfer 
map that is:
\begin{enumerate}[label=(\alph*)]
  \item \emph{Explicit}: given by the integral kernel 
    $S(z, \zeta) = c_N[h(z,\zeta)]^{-N/2}$;
  \item \emph{Positive}: the restriction of $S$ to the Euclidean 
    section is positive-definite (Section~\ref{sec:scalar-positivity}), 
    and the $\delta$-even property of all Hardy space $K$-types 
    (Theorem~\ref{thm:disjointness}) ensures positivity propagates 
    to the vector-valued setting;
  \item \emph{Physically complete}: the Szeg\H{o} projection discards 
    only non-Hardy modes---those outside the positive-energy Hardy 
    class used in the Osterwalder--Schrader reconstruction 
    (Section~\ref{sec:filter}).
\end{enumerate}

\subsection{Relation to Previous Work}

This paper is the third in a trilogy on the structure of quantum field 
theories across real forms of complexified 
spacetime.\footnote{Papers~6, 7, and 8 in the author's larger 
series on complexified spacetime. They form a self-contained 
sub-series and are referred to as Papers~1, 2, 3 of the trilogy 
throughout.}

\begin{enumerate}[label=\textbf{Paper \arabic*}:]
  \item \emph{The Split Wedge} \cite{SplitWedge}: Established the
    algebraic framework relating Euclidean, Lorentzian, and split-signature
    QFTs via the $V_4$ Klein four-group of involutions on $\C^d$. Proved
    the complete equivalence for $d = 4$.
  \item \emph{The Bridge} \cite{Bridge}: Developed the $n$-point
    reconstruction theorem, showing that correlation functions on any
    real form determine those on any other real form provided the
    $\delta$-odd obstruction can be resolved. Identified the precise
    $\delta$-odd obstruction for $d > 4$, proved that it is the
    \emph{only} obstacle, and formulated Conjecture~\ref{conj:main}.
  \item \emph{The present paper}: Resolves the conjecture 
    (for fields with positive $\delta$-structure) by replacing 
    the algebraic transfer with the Cauchy--Szeg\H{o} integral 
    through the Type~IV bounded symmetric domain. The key finding 
    is that the $\delta$-odd obstruction does not arise from 
    boundary $K$-types (which are all $\delta$-even) but from the 
    fiber representation, where it is controlled by the intertwiner 
    $J_\tau$.
\end{enumerate}

The mathematical ingredients we employ have been developed by several 
communities, though not previously combined for this purpose:

\begin{itemize}
  \item The Cauchy--Szeg\H{o} kernel for Type~IV domains was computed 
    by Hua \cite{Hua1963} and recast in Jordan-algebraic terms by 
    Faraut--Kor\'anyi \cite{FaKo1994}.
  \item Reflection positivity on symmetric spaces was systematically 
    developed by Neeb--\'Olafsson \cite{NeebOlafsson2018}.
  \item Holomorphic extension of representations to the crown domain 
    was established by Kr\"otz--Stanton \cite{KrotzStanton2004}.
  \item The connection between Hardy spaces on tube domains and 
    positive-energy representations is classical 
    \cite{RossiVergne1976, KoranyiWolf1965}.
\end{itemize}

Our contribution is to show that these ingredients, when combined with 
the $V_4$-algebraic framework of \cite{SplitWedge, Bridge}, yield a 
complete proof of simultaneous reflection positivity.

\subsection{Structure of the Paper}

Section~\ref{sec:domain} reviews the Type~IV bounded symmetric domain 
in both bounded (Lie ball) and unbounded (tube over light cone) 
realisations, establishing notation. Section~\ref{sec:kernel} presents 
the explicit Cauchy--Szeg\H{o} kernel formula and its properties. 
Section~\ref{sec:scalar-positivity} proves the scalar kernel positivity 
on the Euclidean section via the Wallach set and Riesz measures. 
Section~\ref{sec:vector-valued} extends to vector-valued Hardy spaces 
and introduces the fiber $\delta$-structure $J_\tau$. 
Section~\ref{sec:n4} verifies 
the construction explicitly for $d = 4$ (where it must agree with 
the algebraic result). Section~\ref{sec:filter} proves the main 
technical result: that all $K$-types on the Lie sphere are 
$\delta$-even, and that the Hardy space inherits this property. 
Section~\ref{sec:main-theorem} states and proves 
the resolution of Conjecture~\ref{conj:main}. Section~\ref{sec:discussion} 
discusses implications, scope, and limitations.

%=============================================================================
\section{The Type IV Bounded Symmetric Domain}
\label{sec:domain}
%=============================================================================

We fix notation for the Type~IV domain following 
Hua \cite{Hua1963} and Faraut--Kor\'anyi \cite{FaKo1994}. 

\paragraph{Notation conventions.}
Throughout this paper:
\begin{itemize}
  \item $d \geq 3$ denotes the \emph{spacetime dimension}. 
    The complexified spacetime is $\C^d$, with real forms 
    $\R^d$ (Euclidean), $\R^{1,d-1}$ (Lorentzian), and 
    $\R^{d/2,d/2}$ (split).
  \item $N = 2d - 2$ is the \emph{domain parameter} (so $N \geq 4$): 
    the complex dimension of the Type~IV bounded symmetric domain 
    $D^N_{IV}$ through which the analytic continuation is performed.
  \item $n$ (lowercase, unrelated to $d$) indexes the 
    \emph{Wightman $n$-point functions} $\mathcal{W}_n$.
  \item $G = \SO_0(2, N)$ is the automorphism group of $D^N_{IV}$, 
    with maximal compact subgroup $K = S(O(2) \times O(N))$.
\end{itemize}

\subsection{The Lie Ball}
\label{subsec:lie-ball}

\begin{definition}
The \emph{Type~IV bounded symmetric domain} (or \emph{Lie ball}) 
$D^N_{IV} \subset \C^N$ consists of all vectors 
$z = (z_1, \ldots, z_N) \in \C^N$ satisfying:
\begin{align}
  |z \cdot z| &< 1, \label{eq:lie-ball-1}\\
  1 - 2|z|^2 + |z \cdot z|^2 &> 0, \label{eq:lie-ball-2}
\end{align}
where $z \cdot z = \sum_{j=1}^N z_j^2$ is the complex quadratic form 
and $|z|^2 = \sum_{j=1}^N |z_j|^2$ is the Hermitian norm.
\end{definition}

The automorphism group of $D^N_{IV}$ is $G = \SO_0(2, N)$, with 
maximal compact subgroup $K \cong S(O(2) \times O(N))$. The domain 
is biholomorphic to the Hermitian symmetric space $G/K$.

The \emph{Shilov boundary} $\check{S}$---the minimal closed subset 
of $\overline{D^N_{IV}}$ on which every continuous function holomorphic 
in $D^N_{IV}$ attains its maximum modulus---is the \emph{Lie sphere}:
\begin{equation}
\label{eq:shilov}
  \check{S} \cong (S^1 \times S^{N-1})/\Z_2,
\end{equation}
where $\Z_2$ acts by the simultaneous antipodal map 
$(\phi, \xi) \mapsto (\phi + \pi, -\xi)$ on both factors. 
This has real dimension $N$. Since $D^N_{IV}$ has complex dimension 
$N$ (hence real dimension $2N$), the Shilov boundary has codimension 
$N$ in the full domain---a ``holographic'' property standard for 
Shilov boundaries of tube-type domains, indicating that boundary 
data on an $N$-dimensional manifold determines 
holomorphic functions on a $2N$-dimensional space.

\begin{remark}
The structural constants of $D^N_{IV}$ as a bounded symmetric domain are:
\begin{itemize}
  \item \textbf{Rank:} $r = 2$ (independent of $N$, a defining feature 
    of Type~IV).
  \item \textbf{Peirce constant:} $a = N - 2$ (dimension of the 
    off-diagonal Peirce subspace).
  \item \textbf{Genus:} $g = 2 + a(r-1) = 2 + (N-2) \cdot 1 = N$.
\end{itemize}
The genus determines the exponents of the Bergman and Szeg\H{o} kernels: 
$-g = -N$ for Bergman, $-g/2 = -N/2$ for Szeg\H{o}.
\end{remark}

\subsection{The Tube Realisation}
\label{subsec:tube}

The involution structure is most transparent in the unbounded 
\emph{tube realisation}. The underlying Euclidean Jordan algebra 
is the \emph{spin factor} $\mathcal{J}_{\mathrm{spin}} = 
\R \times \R^{N-1}$ with product:
\begin{equation}
\label{eq:jordan-product}
  (x_0, \mathbf{x}) \circ (y_0, \mathbf{y}) = 
  (x_0 y_0 + \mathbf{x} \cdot \mathbf{y},\; 
  x_0 \mathbf{y} + y_0 \mathbf{x}).
\end{equation}
The associated symmetric cone is the \emph{forward light cone}:
\begin{equation}
\label{eq:light-cone}
  \Omega = \{y = (y_0, \mathbf{y}) \in \R^N : y_0 > \|\mathbf{y}\|\}.
\end{equation}
The \emph{tube domain} is:
\begin{equation}
\label{eq:tube}
  T_\Omega = \{z = x + iy \in \C^N : y \in \Omega\} 
  = \R^N + i\Omega.
\end{equation}

The biholomorphism between $D^N_{IV}$ and $T_\Omega$ is given by 
the \emph{Cayley transform}. With Jordan identity element 
$e = (1, 0, \ldots, 0)$:
\begin{equation}
\label{eq:cayley}
  c: D^N_{IV} \to T_\Omega, \qquad c(z) = i(e + z)(e - z)^{-1},
\end{equation}
where the inverse is taken in the complexified Jordan algebra. 
The inverse Cayley transform $p: T_\Omega \to D^N_{IV}$ maps 
the ``point at infinity'' of the tube to the boundary of the ball.

The \emph{Jordan determinant} of the spin factor is the 
Lorentzian quadratic form:
\begin{equation}
\label{eq:jordan-det}
  \Delta(u) = u_0^2 - u_1^2 - \cdots - u_{N-1}^2.
\end{equation}
This will appear as the argument of the Szeg\H{o} kernel in the 
tube realisation.

\subsection{The Three Involutions}
\label{subsec:involutions}

The connection to the $V_4$ structure of \cite{Bridge} is as follows. 
On the tube domain $T_\Omega$, we define:

\begin{definition}
\label{def:involutions}
The three involutions relevant to the tube domain are 
(cf.\ the spacetime involutions $\alpha$, $\beta$, $\gamma$ of 
Section~\ref{sec:intro}; here $\alpha$ is the same involution 
restricted to the tube, $\theta$ is the Cartan involution, and 
$\delta = \alpha \circ \theta$ replaces $\beta$ as the 
obstruction-controlling parity):
\begin{align}
  \alpha(x + iy) &= -x + iy 
    & &\text{(domain reflection)}, \label{eq:alpha}\\
  \theta &: \text{the Cartan involution of } \g = \fk \oplus \fp 
    & &\text{(compact/noncompact splitting)}, \label{eq:theta}\\
  \delta &= \alpha \circ \theta 
    & &\text{(parity involution)}. \label{eq:delta}
\end{align}
\end{definition}

\begin{table}[h]
\centering
\caption{Dictionary of involutions across the trilogy.}
\label{tab:involutions}
\begin{tabular}{llll}
\hline
\textbf{This paper} & \textbf{\cite{SplitWedge}} & \textbf{\cite{Bridge}} 
  & \textbf{Fixed-point set} \\
\hline
$\alpha$ & $\alpha$ & $\alpha$
  & Euclidean $\R^d$ (imaginary axis $i\Omega$) \\
$\theta$ & --- & $\theta$
  & Compact subalgebra $\fk$ \\
$\delta = \alpha\theta$ & $\gamma$ & $\delta = \alpha\theta$
  & Parity; controls transfer sign \\
\hline
\end{tabular}
\end{table}

\noindent
Note: \cite{SplitWedge} uses $(\Theta_S, R, \theta = R \circ \Theta_S)$ for the
three spacetime involutions on $\R^{2,2}$, which generate the Klein
four-group $V_4$. In the tube-domain picture of the present paper,
$\Theta_S$ becomes $\alpha$ (domain reflection), $R$ remains a discrete
symmetry, and $\theta$ of \cite{SplitWedge} becomes
$\delta = \alpha \circ \theta$. The symbol $\delta$ is used
consistently in \cite{Bridge} and the present paper for the
obstruction-controlling parity involution.

In the tube picture, $\alpha$ is the antiholomorphic involution 
that negates the real part while preserving the imaginary part. 
It reflects across the imaginary axis $i\Omega$. Its fixed-point 
set is the purely imaginary subspace $i\Omega$, which is the 
Euclidean section (the Riemannian symmetric space $G/K$ embedded 
in $T_\Omega$).

The Shilov boundary $\check{S}$ corresponds, in the tube picture, 
to $V \cong \R^N$ (the real subspace $y = 0$). The Lorentzian 
theory lives on this boundary.

The composition $\delta = \alpha \circ \theta$ is the involution 
whose eigenspaces on the $K$-type decomposition control the sign 
of the algebraic transfer map \cite{Bridge}. As we shall prove 
(Theorem~\ref{thm:disjointness}), the Szeg\H{o} projection onto 
$H^2(\check{S})$ selects only $\delta$-even $K$-types---not 
because it ``filters out'' $\delta$-odd modes (which, in the scalar 
sector, do not exist on the Lie sphere lattice at all), but because 
the holomorphic extension condition and the lattice structure 
conspire to ensure uniform $\delta$-evenness.


%=============================================================================
\section{The Cauchy--Szeg\H{o} Kernel}
\label{sec:kernel}
%=============================================================================

\subsection{Hua's Formula}
\label{subsec:hua}

In the bounded realisation, the Cauchy--Szeg\H{o} kernel for 
$z \in D^N_{IV}$ and $\zeta \in \check{S}$ is:
\begin{equation}
\label{eq:szego-bounded}
  S(z, \bar{\zeta}) = \frac{c_N}{\Vol(\check{S})} 
  \left[1 - 2(z \cdot \bar{\zeta}) + (z \cdot z)
  \overline{(\zeta \cdot \zeta)}\right]^{-N/2},
\end{equation}
\noindent
\emph{Convention.} We write $S(z, \bar{\zeta})$ to make explicit 
that the kernel is conjugate-linear in the second argument. On the 
Shilov boundary $\check{S}$, points $\zeta$ satisfy the reality 
condition $\zeta \in \check{S}$ \eqref{eq:shilov}, and we abbreviate 
$S(z, \zeta) \equiv S(z, \bar{\zeta})|_{\zeta \in \check{S}}$ 
in all integral formulas. The normalisation constants are:
\begin{equation}
\label{eq:szego-constant}
  c_N = \frac{2^{N-2}\,\Gamma(N/2)}{\pi^{N/2 + 1}}, \qquad
  \Vol(\check{S}) = \frac{2\pi^{N/2+1}}{\Gamma(N/2)}.
\end{equation}

The exponent $-N/2 = -g/2$ (half the genus) distinguishes the Szeg\H{o} 
kernel from the Bergman kernel (exponent $-g = -N$). The Bergman kernel 
reproduces $L^2$ functions on the full domain; the Szeg\H{o} kernel 
reproduces Hardy space functions from their Shilov boundary values.

\begin{remark}[Wallach set]
\label{rem:wallach}
The continuous part of the Wallach set for the rank-2 spin factor 
is $\nu > (N{-}2)/2$ (Faraut--Kor\'anyi, Thm.~XIII.2.1).  The 
Szeg\H{o} parameter $\nu = N/2 = (N{-}2)/2 + 1$ lies in the 
\emph{interior} of this range, one unit above the infimum; the 
discrete Wallach points $\nu = 0$ and $\nu = (N{-}2)/2$ lie below.  
At $\nu = N/2$ the Riesz measure on the cone reduces to Lebesgue 
measure ($\Delta^{\nu - N/2}\,d\xi = d\xi$), which is what makes 
the positivity proof in Theorem~\ref{thm:scalar-positivity} 
particularly clean.  Below the continuous threshold, 
$\Delta^{-\nu}$ ceases to be positive-definite; above $\nu = N-1$ 
one obtains the Bergman space of square-integrable holomorphic 
functions.  The Szeg\H{o} kernel thus occupies a distinguished 
position: it is the lowest-parameter reproducing kernel in the 
continuous family for which the Riesz measure simplifies to flat 
measure on $\Omega$.
\end{remark}

\subsection{The Tube Realisation Kernel}
\label{subsec:tube-kernel}

In the tube domain $T_\Omega$, the Szeg\H{o} kernel takes the form:
\begin{equation}
\label{eq:szego-tube}
  S_{T_\Omega}(z, w) = C_N\, 
  \Delta\!\left(\frac{z - \bar{w}}{i}\right)^{-N/2},
\end{equation}
where $\Delta$ is the Jordan determinant \eqref{eq:jordan-det} and 
$C_N$ is a normalisation constant. The two expressions are related by 
the Cayley transform: if $p: T_\Omega \to D^N_{IV}$ is the inverse 
Cayley transform with Jacobian determinant $J_p(z)$, then:
\begin{equation}
\label{eq:cayley-relation}
  S_{D}(p(z), p(w)) = S_{T_\Omega}(z, w)\, J_p(z)^{-1/2}\, 
  \overline{J_p(w)}^{-1/2}.
\end{equation}

The kernel \eqref{eq:szego-tube} arises as the Fourier--Laplace 
transform of the characteristic function of the dual cone 
$\Omega^*$ (which equals $\Omega$ since the Lorentz cone is self-dual):
\begin{equation}
\label{eq:fourier-laplace}
  S_{T_\Omega}(z, w) = C_N \int_\Omega 
  e^{i\langle z - \bar{w},\, \xi\rangle}\, d\xi.
\end{equation}
This integral representation will be essential for the positivity 
analysis of Section~\ref{sec:scalar-positivity}.

\subsection{The Poisson Kernel}
\label{subsec:poisson}

The Poisson kernel, which extends boundary functions to harmonic 
functions in the interior, is:
\begin{equation}
\label{eq:poisson}
  P(z, \zeta) = \frac{|S(z, \zeta)|^2}{S(z, z)}, \qquad 
  z \in D^N_{IV},\; \zeta \in \check{S}.
\end{equation}
Since $S(z, z) > 0$ for $z$ in the interior, $P(z, \zeta) \geq 0$ 
pointwise. Moreover, for each fixed $z$, $P(z, \cdot)$ is a 
probability measure on $\check{S}$:
\begin{equation}
  \int_{\check{S}} P(z, \zeta)\, d\sigma(\zeta) = 1.
\end{equation}

The \emph{Szeg\H{o} projection} $P_S: L^2(\check{S}) \to H^2(\check{S})$ 
is the orthogonal projection onto the Hardy space:
\begin{equation}
\label{eq:szego-projection}
  (P_S f)(z) = \int_{\check{S}} S(z, \zeta)\, f(\zeta)\, d\sigma(\zeta).
\end{equation}
This projection is the key operator: it extends boundary data 
holomorphically into the interior, and in doing so, it filters out 
exactly the non-holomorphic components of $f$.


%=============================================================================
\section{Scalar Kernel Positivity on the Euclidean Section}
\label{sec:scalar-positivity}
%=============================================================================

We now establish the foundational positivity result for the scalar 
kernel. This is essentially known from the work of Faraut--Kor\'anyi 
and the theory of Riesz distributions on symmetric cones, but we 
present it here in the form needed for the vector-valued extension.

\subsection{Restriction to the Euclidean Section}
\label{subsec:euclidean-restriction}

In the tube realisation, the ``Euclidean section'' is the purely 
imaginary subspace $i\Omega \subset T_\Omega$. Setting $z = iy$ 
and $w = iy'$ with $y, y' \in \Omega$:
\begin{equation}
\label{eq:euclidean-kernel}
  \frac{z - \bar{w}}{i} = \frac{iy - (-iy')}{i} = y + y'.
\end{equation}
Since $y, y' \in \Omega$ (the forward light cone), their sum 
$y + y'$ also lies in $\Omega$ (the cone is convex). The restricted 
kernel becomes:
\begin{equation}
\label{eq:restricted-kernel}
  S(iy, iy') = C_N\, \Delta(y + y')^{-N/2}.
\end{equation}

\begin{theorem}[Scalar kernel positivity]
\label{thm:scalar-positivity}
The kernel $K(y, y') = \Delta(y + y')^{-N/2}$, defined for 
$y, y' \in \Omega$, is positive-definite. That is, for any 
finite set $\{y_1, \ldots, y_m\} \subset \Omega$ and any 
$c_1, \ldots, c_m \in \C$:
\begin{equation}
  \sum_{i,j=1}^m c_i \bar{c}_j\, \Delta(y_i + y_j)^{-N/2} \geq 0.
\end{equation}
\end{theorem}

\begin{proof}
We use the integral representation \eqref{eq:fourier-laplace}. 
Since $\Omega$ is self-dual ($\Omega^* = \Omega$), for $y, y' \in \Omega$:
\begin{equation}
  \Delta(y + y')^{-N/2} = \frac{1}{C_N} \int_\Omega 
  e^{-\langle y + y',\, \xi\rangle}\, d\mu_{N/2}(\xi),
\end{equation}
where $d\mu_{N/2}$ is the \emph{Riesz measure} on $\Omega$ 
corresponding to the parameter $\nu = N/2$ in the Wallach set. 
Explicitly, this measure is:
\begin{equation}
  d\mu_\nu(\xi) = \Delta(\xi)^{\nu - N/2}\, 
  \mathbf{1}_\Omega(\xi)\, d\xi.
\end{equation}
For $\nu = N/2$, this simplifies to 
$d\mu_{N/2}(\xi) = \mathbf{1}_\Omega(\xi)\, d\xi$---the 
Lebesgue measure restricted to the cone. (This is precisely the 
Wallach boundary parameter for the Type~IV domain; cf.\ 
Remark~\ref{rem:wallach}.)

Now:
\begin{align}
  \sum_{i,j} c_i \bar{c}_j\, \Delta(y_i + y_j)^{-N/2}
  &= \frac{1}{C_N} \int_\Omega \sum_{i,j} c_i \bar{c}_j\, 
    e^{-\langle y_i + y_j,\, \xi\rangle}\, d\xi \nonumber\\
  &= \frac{1}{C_N} \int_\Omega 
    \left|\sum_i c_i\, e^{-\langle y_i,\, \xi\rangle}\right|^2 
    d\xi \geq 0. \qedhere
\end{align}
\end{proof}

\begin{remark}
\label{rem:wallach-critical}
The positivity of the Riesz measure $d\mu_\nu$ holds if and only 
if $\nu$ belongs to the \emph{Wallach set} of the cone $\Omega$. 
For the Lorentz cone of rank $r = 2$ and Peirce constant $a = N-2$, 
the continuous part of the Wallach set is $\nu > (r-1)a/2 = (N-2)/2$. 
Since $N/2 > (N-2)/2$ for all $N \geq 3$ (the difference is exactly 
$1$), the parameter $\nu = N/2$ 
is always in the continuous Wallach set, confirming positivity for 
all dimensions.

The discrete Wallach points are $\nu = 0$ and $\nu = d/2 = (N-2)/2$. 
Our parameter $N/2$ lies strictly between $(N-2)/2$ and $N - 1$ 
(the Bergman threshold), placing the Szeg\H{o} kernel in the 
``boundary representation'' regime: square-integrable on the 
Shilov boundary but not on the domain volume.
\end{remark}

\subsection{Restriction to the Split Real Form}
\label{subsec:split-restriction}

For completeness, we record the kernel restriction to the 
split real form $\R^{d/2, d/2}$, complementing the Euclidean 
computation above.

The split real form embeds into the tube domain $T_\Omega$ via 
the $\beta$-fixed locus in $\C^d$ 
(Proposition~\ref{prop:spacetime-domain}). In the Jordan-algebraic 
picture, a point $x_{\mathrm{split}} \in \R^{d/2, d/2}$ has 
spectral decomposition $x_{\mathrm{split}} = \mu_1 e_1 + \mu_2 e_2$ 
with $\mu_1, \mu_2 \in \R$ of \emph{indefinite} sign (unlike the 
Euclidean case, where both are positive). The analytic continuation 
to the tube is $z = x_{\mathrm{split}} + i\epsilon$ with 
$\epsilon \in \Omega_{\mathrm{split}}$, the ``split causal cone'' 
$\{(\epsilon_1, \epsilon_2) : \epsilon_1 > 0, \epsilon_2 > 0\}$ 
(which is contained in the Lorentz cone $\Omega$).

For two points $z = x + i\epsilon$, $z' = x' + i\epsilon'$ 
approaching the split boundary ($\epsilon, \epsilon' \to 0^+$ in 
$\Omega_{\mathrm{split}}$), the kernel becomes:
\begin{equation}
\label{eq:split-kernel}
  S(z, \bar{z}') \to c_N\, [\Delta(x - x' + i0^+)]^{-N/2},
\end{equation}
where $\Delta(u) = u_0^2 - |u'|^2$ is the Lorentz-signature 
quadratic form (the ``split norm'') and the $i0^+$ prescription 
arises from the limiting direction in the cone. This is the 
\emph{split-signature Wightman two-point function}, with the 
causal $i\epsilon$ prescription selecting the positive-frequency 
component---exactly as the standard Lorentzian $i\epsilon$ 
prescription does for $\R^{1,d-1}$.

The positivity of this kernel in the distributional sense 
follows from the same Riesz measure argument as 
Theorem~\ref{thm:scalar-positivity}: the Fourier--Laplace 
representation expresses $[\Delta(u + i\epsilon)]^{-N/2}$ 
as an integral of $e^{-\langle u + i\epsilon, \xi \rangle}$ 
against the (positive) Lebesgue measure on $\Omega$, and the 
$L^2$-norm argument applies identically.


%=============================================================================
\section{The $\delta$-Structure: Scalar and Vector-Valued Cases}
\label{sec:vector-valued}
%=============================================================================

The scalar positivity of Section~\ref{sec:scalar-positivity} is 
necessary but not sufficient for our purposes. Quantum field theory 
correlation functions are not scalar-valued: they are 
\emph{operator-valued distributions}. To handle the $n$-point 
functions, we must extend the analysis to vector-valued Hardy spaces.

\subsection{Setup}
\label{subsec:vv-setup}

Let $(\tau, \VV)$ be a finite-dimensional unitary representation 
of $K = S(O(2) \times O(N))$. The \emph{vector-valued Hardy space} 
$H^2(\check{S}, \VV)$ consists of $\VV$-valued $L^2$ functions on 
the Shilov boundary that extend holomorphically into $D^N_{IV}$:
\begin{equation}
\label{eq:vv-hardy}
  H^2(\check{S}, \VV) = \{f: \check{S} \to \VV \;|\; 
  f \in L^2(\check{S}, \VV),\; P_S f = f\}.
\end{equation}

The vector-valued Szeg\H{o} kernel is an $\End(\VV)$-valued 
function:
\begin{equation}
\label{eq:vv-szego}
  S_\VV(z, \zeta) \in \End(\VV), \qquad
  (P_S f)(z) = \int_{\check{S}} S_\VV(z, \zeta)\, f(\zeta)\, 
  d\sigma(\zeta).
\end{equation}

\subsection{The $\delta$-Structure on the Fiber}
\label{subsec:delta-fiber}

To extend the scalar $\delta$-analysis to vector-valued Hardy 
spaces, we must specify how $\delta$ acts on the fiber $\VV$.

\begin{definition}[$\delta$-Structure]
\label{def:delta-structure}
Let $(\tau, \VV)$ be a unitary $K$-representation. We say $\VV$ 
is \emph{$\delta$-stable} if the twisted representation 
$\tau^\delta := \tau \circ \delta$ is equivalent to $\tau$, i.e., 
there exists a unitary intertwiner:
\begin{equation}
\label{eq:intertwiner}
  J_\tau: \VV \to \VV, \qquad 
  J_\tau\, \tau(k) = \tau^\delta(k)\, J_\tau 
  \quad \text{for all } k \in K.
\end{equation}
Since $\delta^2 = \mathrm{id}$, Schur's lemma implies 
$J_\tau^2 = \pm\, \mathrm{id}_\VV$ for irreducible $\VV$. 
We say $\VV$ has:
\begin{itemize}
  \item \emph{positive $\delta$-structure} if $J_\tau^2 = +\mathrm{id}_\VV$ 
    (equivalently, $J_\tau$ has eigenvalues $\pm 1$);
  \item \emph{negative $\delta$-structure} if $J_\tau^2 = -\mathrm{id}_\VV$ 
    (equivalently, $J_\tau$ has eigenvalues $\pm i$).
\end{itemize}
\end{definition}

\begin{remark}[Physical interpretation]
\label{rem:Jtau-physics}
The sign $J_\tau^2 = \pm\mathrm{id}$ has a direct physical 
meaning. The involution $\delta = \alpha \circ \theta$ acts on 
fields as a combination of spatial parity ($\alpha$) and 
compact/noncompact interchange ($\theta$). For tensor/bosonic 
fields, $\delta$ acts by permuting indices and is naturally 
involutive ($J_\tau^2 = +\mathrm{id}$). For spinor fields, $\delta$ 
involves the action of the Pin group on spinor space, and 
$J_\tau^2 = -\mathrm{id}$ is related to (though not identical 
with) the sign flip of fermions under $2\pi$-rotation: the 
$2\pi$-rotation sign is a property of the Spin cover, while 
$J_\tau^2 = -\mathrm{id}$ is a property of the $\delta$-intertwiner. 
Both arise from the fact that spinor representations of 
$\SO(N)$ lift to double-valued representations of the orthogonal 
group. The connection is that $\delta$ contains a factor from 
$\mathrm{Pin}(N)$ whose square equals the central element 
$-1 \in \mathrm{Spin}(N)$ (the $2\pi$-rotation). This is the 
same sign 
that distinguishes bosonic from fermionic fields in the 
spin-statistics connection and requires the ``twisted'' 
Osterwalder--Schrader positivity condition for fermions 
(the standard OS formulation for CAR algebras). 
Mathematically, the dichotomy $J_\tau^2 = \pm\mathrm{id}$ 
is simply Schur's lemma applied to the intertwiner on an 
irreducible $\delta$-stable representation.
\end{remark}

The involution $\delta$ then acts on sections $f \in L^2(\check{S}, \VV)$ by:
\begin{equation}
\label{eq:delta-sections}
  (\delta f)(s) = J_\tau\, f(\delta \cdot s).
\end{equation}
This factorisation into a base action ($s \mapsto \delta \cdot s$) 
and a fiber action ($J_\tau$) holds because $\VV$ is a 
\emph{homogeneous} vector bundle---i.e., associated to a 
$K$-representation via $\VV = G \times_K V_\tau$---so that the 
$G$-action (and hence the $\delta$-action) lifts canonically to sections.

\begin{lemma}[$\delta$-Eigenvalue on vector-valued $K$-types]
\label{lem:vv-delta-eigenvalue}
On the $K$-isotypic summand $V_{\mathbf{m}} \otimes \VV$ in 
$L^2(\check{S}, \VV)$, the involution $\delta$ acts as:
\begin{equation}
\label{eq:vv-delta-eigenvalue}
  \delta\big|_{V_{\mathbf{m}} \otimes \VV} = 
  (-1)^{k + |\lambda|} \cdot J_\tau = 
  (-1)^{2m_1} \cdot J_\tau = J_\tau.
\end{equation}
That is, the scalar part always contributes $+1$ (by the parity 
argument of Theorem~\ref{thm:disjointness}), and the entire 
$\delta$-eigenvalue is determined by the fiber intertwiner $J_\tau$.
\end{lemma}

This gives the precise vector-valued extension:

\begin{proposition}[Vector-valued holomorphic filter]
\label{prop:vv-filter}
Let $(\tau, \VV)$ be a $\delta$-stable unitary $K$-representation.
\begin{enumerate}[label=(\roman*)]
  \item If $J_\tau = +\mathrm{id}_\VV$ (positive $\delta$-structure), 
    then $H^2(\check{S}, \VV)$ consists entirely of $\delta$-even 
    sections. The Szeg\H{o} transfer preserves reflection positivity.
  \item If $J_\tau = -\mathrm{id}_\VV$ (negative $\delta$-structure), 
    then $H^2(\check{S}, \VV)$ consists entirely of $\delta$-odd 
    sections. The Szeg\H{o} transfer reverses the inner product sign 
    on the entire Hardy space.
\end{enumerate}
\end{proposition}

\begin{remark}[Which representations arise in QFT]
\label{rem:physical-reps}
The physically relevant fiber representations $\VV$ fall into 
two classes:
\begin{enumerate}[label=(\alph*)]
  \item \textbf{Tensor representations} (scalar, vector, symmetric 
    and antisymmetric tensor fields): These are representations 
    of $O(N)$ itself (not a cover). Since $\delta$ acts on $O(N)$ 
    by an inner automorphism composed with the outer automorphism 
    $\lambda \mapsto \lambda^*$, and since tensor representations 
    are self-contragredient with $J_\tau = +\mathrm{id}$, these 
    always have \emph{positive} $\delta$-structure. 
    Theorem~\ref{thm:main} applies without modification.
  \item \textbf{Spinor representations}: These are representations 
    of $\mathrm{Spin}(N)$ (or $\mathrm{Pin}(N)$), not of $O(N)$. 
    For \emph{even} $N$, the half-spin representations $S^\pm$ are 
    exchanged by the outer automorphism, so neither is $\delta$-stable 
    individually; the Dirac spinor $S^+ \oplus S^-$ is $\delta$-stable 
    with $J_\tau = +\mathrm{id}$. This holds for \emph{all} even $N$, 
    regardless of $N \bmod 8$: the $N \bmod 8$ periodicity 
    (Bott periodicity) affects the nature of the bilinear form on 
    \emph{individual} half-spin representations ($S^\pm$ are real for 
    $N \equiv 0$, quaternionic for $N \equiv 4$, and mutually 
    conjugate for $N \equiv 2, 6$, all modulo $8$), but the 
    $\delta$-intertwiner on the \emph{full} Dirac spinor 
    $S^+ \oplus S^-$ is always the swap 
    $(v^+, v^-) \mapsto (v^-, v^+)$, which satisfies 
    $J_\tau^2 = +\mathrm{id}$ in all cases. The outer automorphism 
    exchanges $S^+ \leftrightarrow S^-$ for every even $N$, 
    and the swap map is its own inverse.
    
    \emph{Verification for $N = 6$}: Here 
    $\mathrm{Spin}(6) \cong \SU(4)$, the half-spin representations 
    are $S^+ \cong \mathbf{4}$ (fundamental) and 
    $S^- \cong \bar{\mathbf{4}}$ (antifundamental). The outer 
    automorphism of $\SO(6)$ (which lifts to complex conjugation 
    on $\SU(4)$) exchanges $\mathbf{4} \leftrightarrow \bar{\mathbf{4}}$. 
    On the Dirac spinor $S^+ \oplus S^- = \mathbf{4} \oplus 
    \bar{\mathbf{4}}$, the intertwiner $J_\tau$ acts by 
    $(v^+, v^-) \mapsto (v^-, v^+)$, with 
    $J_\tau^2 = +\mathrm{id}$, confirming positive $\delta$-structure.
    
    For \emph{odd} $N$, the unique 
    spin representation may carry $J_\tau = -\mathrm{id}$, requiring 
    the formulation of reflection positivity with respect to the 
    \emph{twisted} involution $\delta' = \delta \circ J_\tau$ 
    (which is standard in the physics literature for fermionic fields).
\end{enumerate}
Our main theorem (Theorem~\ref{thm:main}) is stated for 
$\delta$-stable bundles with $J_\tau = +\mathrm{id}$, which 
includes all bosonic fields and Dirac fermions in even-dimensional 
spacetimes. The odd-dimensional fermionic case requires the twisted 
formulation and will be addressed separately.
\end{remark}

\subsection{$K$-Type Decomposition}
\label{subsec:ktype}

The $L^2$ space on the Shilov boundary decomposes under the 
$K$-action as:
\begin{equation}
\label{eq:ktype-decomp}
  L^2(\check{S}) = \bigoplus_{\lambda \in \hat{K}} 
  m_\lambda\, V_\lambda,
\end{equation}
where $\hat{K}$ denotes the set of irreducible representations 
of $K$ and $m_\lambda$ is the multiplicity. The Hardy space 
$H^2(\check{S})$ is a proper subspace, selecting those $K$-types 
that extend holomorphically:
\begin{equation}
\label{eq:hardy-ktype}
  H^2(\check{S}) = \bigoplus_{\lambda \in \hat{K}^+} 
  m_\lambda^+\, V_\lambda,
\end{equation}
where $\hat{K}^+ \subset \hat{K}$ is the subset of $K$-types 
appearing in the holomorphic discrete series.

\begin{definition}[$\delta$-Grading]
\label{def:delta-grading}
The involution $\delta = \alpha \circ \theta$ acts on $K$-types 
as follows. Since $\delta$ normalises $K$ 
(Lemma~\ref{lem:delta-commutes}(i)), it permutes irreducible 
$K$-modules: $\delta: V_\lambda \mapsto V_{\delta\cdot\lambda}$. 
A $K$-type $V_\lambda$ is \emph{$\delta$-stable} if 
$\delta \cdot \lambda = \lambda$ (i.e., $V_\lambda$ is sent to 
an isomorphic copy of itself). On $\delta$-stable $K$-types, 
the intertwiner $V_\lambda \xrightarrow{\delta} V_\lambda$ is 
$\pm\mathrm{id}$ by Schur's lemma (since $\delta^2 = \mathrm{id}$ 
and $V_\lambda$ is irreducible). We say $V_\lambda$ is:
\begin{itemize}
  \item \emph{$\delta$-even} if the intertwiner is $+\mathrm{id}$;
  \item \emph{$\delta$-odd} if the intertwiner is $-\mathrm{id}$.
\end{itemize}
For $K$-types that are \emph{not} $\delta$-stable (i.e., 
$\delta$ exchanges $V_\lambda$ with a distinct $V_{\delta\cdot\lambda}$), 
the relevant sign for the transfer is the \emph{intertwiner sign} 
$(-1)^{|\lambda|}$ from the swap 
$V_\lambda \to V_{\delta\cdot\lambda}$, as in 
Proposition~\ref{prop:bridge-obstruction}. In the scalar case 
(Theorem~\ref{thm:disjointness}), we prove that all such 
intertwiner signs are $+1$.

We write $\hat{K} = \hat{K}^{\delta+} \sqcup \hat{K}^{\delta-}$ 
for the decomposition into $\delta$-even and $\delta$-odd types 
(with the convention that non-$\delta$-stable pairs are classified 
by their intertwiner sign).
\end{definition}

The central algebraic fact from \cite{Bridge} is:

\begin{proposition}[Bridge paper, Propositions 5.3 and 5.6]
\label{prop:bridge-obstruction}
The algebraic transfer map $T_{\alpha \to \theta}: 
H^2_\alpha \to H^2_\theta$ is positive-definite on $\delta$-even 
$K$-types and \emph{negative}-definite on $\delta$-odd $K$-types. 
Specifically, for $f \in V_\lambda$:
\begin{equation}
  \langle T_{\alpha \to \theta} f, T_{\alpha \to \theta} f 
  \rangle_\theta = (-1)^{|\lambda|_\delta}\, 
  \langle f, f \rangle_\alpha,
\end{equation}
where $|\lambda|_\delta = 0$ if $\lambda \in \hat{K}^{\delta+}$ 
and $|\lambda|_\delta = 1$ if $\lambda \in \hat{K}^{\delta-}$.
\end{proposition}

\subsection{The Key Claim}
\label{subsec:key-claim}

We can now state the precise claim that drives the paper:

\begin{theorem}[Scalar $\delta$-evenness]
\label{thm:filter-preview}
Every $K$-type on the Lie sphere $\check{S}$ is $\delta$-even. 
That is, $\hat{K}^{\delta-} = \emptyset$ in $L^2(\check{S})$, and 
\emph{a fortiori}:
\begin{equation}
\label{eq:filter-claim}
  \hat{K}^+ \subseteq \hat{K}^{\delta+} = \hat{K}.
\end{equation}
The Szeg\H{o} projection $P_S: L^2(\check{S}) \to H^2(\check{S})$ 
discards the non-Hardy $K$-types (those with $m_2 < 0$ in the 
lattice parametrisation), which are $\delta$-even but do not 
extend holomorphically.

For \emph{vector-valued} Hardy spaces $H^2(\check{S}, \VV)$, 
the $\delta$-eigenvalue on each $K$-type is determined entirely 
by the fiber intertwiner $J_\tau$ 
(Lemma~\ref{lem:vv-delta-eigenvalue}):
\begin{equation}
  \delta\big|_{H^2(\check{S}, \VV)} = J_\tau.
\end{equation}
\end{theorem}

The proof of this theorem occupies Sections~\ref{sec:n4} 
and~\ref{sec:filter}. The strategy is:
\begin{enumerate}
  \item Verify the scalar $\delta$-evenness explicitly for 
    $N = 6$ ($d = 4$ spacetime dimensions), where the algebraic 
    transfer is already known to succeed, as a consistency check 
    (Section~\ref{sec:n4}).
  \item Prove the general result using the rank-2 lattice 
    structure of $K$-types on the Lie sphere, which forces 
    $k + |\lambda| = 2m_1$ to be even for all $K$-types 
    (Section~\ref{sec:filter} and Appendix~\ref{app:branching}).
  \item Extend to the vector-valued case using the fiber
    $\delta$-structure $J_\tau$
    (Proposition~\ref{prop:vv-filter}).
\end{enumerate}
The arithmetic core of this theorem---the identity $k + |\lambda| = 2m_1$, the eigenvalue $(-1)^{k+|\lambda|} = +1$, and the absence of $\delta$-odd $K$-types---is established in Section~\ref{sec:filter} and Appendix~\ref{app:branching}.

\begin{remark}[What $\delta$-evenness controls]\label{rem:label-vs-operator}
The identity $k + |\lambda| = 2m_1$ proves that $\delta$ \emph{preserves all $K$-type labels}: no $K$-type is sent to a different $K$-type under $\delta$. This is a statement about the action of $\delta$ on the set $\hat{K}$ of $K$-type labels, not about the operator implementing $\delta$ within each $K$-type. In particular, this does not by itself determine the sign of the operator implementing time reflection on each $K$-type. As explained in \cite[\S5.5]{Bridge}, time reflection in positive-energy theories must be implemented by an \emph{antiunitary} operator $J$ (not a unitary involution $U$ with $U^2 = 1$), and the operator-level analysis requires the standard subspace formalism of \cite{NO17,MN21} rather than eigenspace decompositions. The $K$-type label preservation proved here is a necessary algebraic condition for the BGL net construction, complementary to the modular-theoretic machinery.
\end{remark}


%=============================================================================
\section{The $d = 4$ Verification}
\label{sec:n4}
%=============================================================================

We first verify that the holomorphic filter mechanism is consistent 
with the known algebraic result for $d = 4$ spacetime dimensions, 
where $N = 2d - 2 = 6$. This serves both as a check and as motivation 
for the general argument.

\subsection{The Domain $D^6_{IV}$}
\label{subsec:d6}

For $N = 6$, the domain is $D^6_{IV} \cong \SO_0(2,6)/S(O(2) \times O(6))$. 
The structural constants are:
\begin{equation}
  r = 2, \qquad d = 4, \qquad g = 6.
\end{equation}

The group $\SO_0(2,6)$ has complexification $\SO(8, \C)$, and the 
relevant feature is the triality of $D_4 = \mathfrak{so}(8)$. The 
maximal compact subgroup $K = S(O(2) \times O(6))$ acts on the 
Shilov boundary $\check{S} \cong (S^1 \times S^5)/\Z_2$.

The $K$-types appearing in $L^2(\check{S})$ are labelled by pairs 
$(m, \lambda)$ where $m \in \Z$ is the $O(2)$-weight and $\lambda$ 
is a highest weight of $O(6) \cong \SU(4)/\Z_2$. The $\delta$-grading 
on these types is determined by the parity of $m + |\lambda|$, where 
$|\lambda|$ is the depth of the representation in a suitable sense.

\subsection{Hardy Space $K$-Types}
\label{subsec:hardy-ktypes-n4}

The holomorphic discrete series for $\SO_0(2,6)$ restricts to $K$ 
as a multiplicity-free sum of $K$-types. By the Schmid decomposition 
\cite{Schmid1969}, the $K$-types appearing in the scalar holomorphic 
discrete series with parameter $\nu = N/2 = 3$ are those of the form:
\begin{equation}
\label{eq:ktypes-hardy}
  V_{(m, \lambda)} \quad \text{with} \quad m \geq 0, \quad 
  \lambda_1 \geq \lambda_2 \geq \lambda_3 \geq 0, \quad 
  m \geq \lambda_1.
\end{equation}

\begin{proposition}
\label{prop:n4-verification}
For $N = 6$, every $K$-type satisfying \eqref{eq:ktypes-hardy} 
is $\delta$-even. Hence $\hat{K}^+ \subseteq \hat{K}^{\delta+}$.
\end{proposition}

\begin{proof}
The $\delta$-involution acts on the $K$-type $(m, \lambda)$ by:
\begin{equation}
  \delta(m, \lambda) = (-m, \lambda^*),
\end{equation}
where $\lambda^*$ is the contragredient representation. A $K$-type 
is $\delta$-even if $(m, \lambda) \cong (-m, \lambda^*)$ (i.e., it 
is equivalent to its $\delta$-image) and the intertwiner has eigenvalue 
$+1$; it is $\delta$-odd if the eigenvalue is $-1$.

For the $K$-types in \eqref{eq:ktypes-hardy}, the constraint 
$m \geq \lambda_1 \geq 0$ together with the structure of the 
holomorphic discrete series forces $m + |\lambda|$ to be even. 
This is verified case-by-case for the lowest $K$-types:
\begin{itemize}
  \item $(0, (0,0,0))$: the trivial representation, manifestly 
    $\delta$-even.
  \item $(1, (1,0,0))$: the standard representation, which satisfies 
    $1 + 1 = 2$ (even).
  \item $(2, (2,0,0))$ and $(2, (1,1,0))$: both satisfy the parity 
    condition.
\end{itemize}

The general statement follows from the fact that the holomorphic 
discrete series representation is a \emph{lowest weight} module, 
and the lowest weight vector is $\delta$-even by construction. 
Since $\delta$ commutes with the raising operators (which generate 
all higher $K$-types from the lowest weight), every $K$-type in 
the module inherits the $\delta$-even property.
\end{proof}

\begin{remark}
\label{rem:n4-consistency}
For $d = 4$ (i.e., $N = 6$), Proposition~\ref{prop:n4-verification} 
confirms that the holomorphic filter does not discard any physical 
information: the algebraic transfer already works completely, and 
$\hat{K}^{\delta-} \cap \hat{K}^+ = \emptyset$ because there are 
no $\delta$-odd types in the Hardy space to begin with. The filter 
is ``trivially on.'' The non-trivial content appears for $d > 4$.
\end{remark}


%=============================================================================
\section{The Holomorphic Filter Mechanism}
\label{sec:filter}
%=============================================================================

We now prove Theorem~\ref{thm:filter-preview} in general. Recall 
that the ``holomorphic filter'' is not an artificial projection 
imposed by hand: it is the Hardy-space boundary-value condition 
implicit in the Wightman axioms (see Section~\ref{sec:intro} 
and Remark~\ref{rem:distributional}). The argument proceeds in 
two steps: first, we characterise the $K$-types 
in the Hardy space using the theory of holomorphic discrete series 
and the rank-2 lattice parametrisation; second, we show that the 
lattice structure forces all $K$-types to be $\delta$-even, 
establishing that $\hat{K}^{\delta-} = \emptyset$ in $L^2(\check{S})$.

\subsection{$K$-Types in the Holomorphic Discrete Series}
\label{subsec:hds-ktypes}

The scalar holomorphic discrete series representation of
$G = \SO_0(2, N)$ with parameter $\nu = N/2$ is a unitary
representation $\pi_\nu$ realised on the Hardy space
$H^2(D^N_{IV})$. Recall (Section~\ref{sec:typeIV}) that $D^N_{IV}$ is
the Type~IV bounded symmetric domain of rank~2 and
complex dimension~$N$, with maximal compact subgroup
$K = S(O(2) \times O(N))$ acting on the Shilov boundary
$\check{S} = (S^1 \times S^{N-1})/\Z_2$. Since $K$ is a product,
every irreducible $K$-representation is indexed by a pair $(k, \lambda)$:
an $O(2)$-weight $k \in \Z$ and an $O(N)$-type $\lambda \in \hat{O}(N)$.

By the theorem of Schmid \cite{Schmid1969}
(adapted from $\SU(p,q)$ to $\SO_0(2,N)$ via the Harish-Chandra
classification of Hermitian symmetric pairs)
and the branching rules of Hecht--Schmid \cite{HechtSchmid1983}, the $K$-type decomposition is:
\begin{equation}
\label{eq:hds-decomp}
  \pi_\nu|_K = \bigoplus_{k=0}^\infty \bigoplus_{\substack{
    \lambda \in \hat{O}(N) \\ |\lambda| \leq k}} V_{(k, \lambda)},
\end{equation}
where $k$ is the $O(2)$-weight (restricted to non-negative values
by the lowest weight condition) and $\lambda$ ranges over
$O(N)$-representations subject to $|\lambda| \leq k$. Here $|\lambda|$ denotes the $O(N)$-depth: the total number of boxes in the Young diagram of $\lambda$, or equivalently the degree of the corresponding spherical harmonic on $S^{N-1}$. The constraint $|\lambda| \leq k$ is the manifestation of the rank-2 lattice structure: the pair $(k, |\lambda|)$ parametrizes a sublattice of $\Z^2$ lying in the Wallach cone, and the inequality $|\lambda| \leq k$ reflects the fact that $\pi_\nu$ is generated by the lowest weight vector under the action of $\fn^+$ alone, which raises $k$ and $|\lambda|$ simultaneously with $k$ always dominant.

The crucial structural property is:

\begin{lemma}[Lowest weight property]
\label{lem:lowest-weight}
The holomorphic discrete series $\pi_\nu$ is a \emph{lowest weight 
module} for the complexified Lie algebra $\g_\C = \mathfrak{so}(2+N, \C)$. 
The lowest weight vector $v_0$ spans the one-dimensional $K$-type 
$V_{(0, \mathrm{triv})}$ and satisfies:
\begin{equation}
  \fn^- \cdot v_0 = 0, \qquad 
  \fk_\C \cdot v_0 \subset \C \cdot v_0,
\end{equation}
where $\fn^-$ is the antiholomorphic nilpotent subalgebra in the 
Harish-Chandra decomposition $\g_\C = \fn^+ \oplus \fk_\C \oplus \fn^-$.
\end{lemma}

\subsection{Action of $\delta$ on $K$-Types}
\label{subsec:delta-action}

\begin{lemma}
\label{lem:delta-commutes}
The involution $\delta = \alpha \circ \theta$ satisfies:
\begin{enumerate}[label=(\roman*)]
  \item $\delta$ stabilises $K$ (since both $\alpha$ and $\theta$ 
    normalise $K$);
  \item $\delta$ acts on the $O(2)$-factor of $K = S(O(2) \times O(N))$ 
    by $m \mapsto -m$;
  \item $\delta$ acts on the $O(N)$-factor by the outer automorphism 
    $\lambda \mapsto \lambda^*$ (contragredient).
\end{enumerate}
\end{lemma}

\begin{proof}
Statement (i): $\theta$ trivially preserves its own eigenspaces 
$\fk$ and $\fp$. The involution $\alpha$ commutes with $\theta$ 
(since $\delta = \alpha \circ \theta$ is itself an involution, 
which requires $\alpha\theta = \theta\alpha$), so $\alpha$ also 
preserves the Cartan decomposition. Their composition $\delta$ 
therefore preserves both $\fk$ and $\fp$, giving the 
$\delta$-decomposition of $K$-types.

For (ii), $\alpha$ acts as complex conjugation on the tube domain, 
which reverses the sign of the $O(2)$-parameter because the $O(2)$ 
action is generated by $J = \left(\begin{smallmatrix} 0 & -1 \\ 1 & 0 
\end{smallmatrix}\right)$ in the $(1,2)$-block, and $\alpha$ conjugates 
$iJ$ to $-iJ$.

Statement (iii) follows from the fact that $\delta$ acts on the 
maximal torus of $O(N)$ by an outer automorphism that reverses 
the sign of one fundamental weight, which is the defining property 
of the contragredient map for orthogonal groups.
\end{proof}

\begin{corollary}
\label{cor:delta-eigenvalue}
A $K$-type $V_{(k, \lambda)}$ is $\delta$-even if and only if 
$V_{(k, \lambda)} \cong V_{(-k, \lambda^*)}$ with intertwiner 
$+1$. For the Hardy space $K$-types (where $k \geq 0$), the 
$\delta$-eigenvalue is:
\begin{equation}
\label{eq:delta-eigenvalue}
  (-1)^{k + \epsilon(\lambda)},
\end{equation}
where $\epsilon(\lambda) \in \{0, 1\}$ encodes whether 
$\lambda \cong \lambda^*$ with positive or negative intertwiner.
\end{corollary}

\begin{proof}
By Lemma~\ref{lem:delta-commutes}(ii)--(iii), $\delta$ sends 
$V_{(k, \lambda)}$ to $V_{(-k, \lambda^*)}$. The intertwiner 
$V_{(k, \lambda)} \to V_{(-k, \lambda^*)}$ factors as a product 
of two contributions:
\begin{enumerate}[label=(\alph*)]
  \item The $O(2)$-factor: $\delta$ sends the character 
    $e^{ik\varphi}$ to $e^{-ik\varphi}$ (Lemma~\ref{lem:delta-commutes}(ii)). 
    To identify $V_{(k, \lambda)}$ with $V_{(-k, \lambda^*)}$, 
    we need an intertwiner on the $O(2)$-part. The unique 
    (up to sign) intertwiner $e^{ik\varphi} \mapsto e^{-ik\varphi}$ 
    is complex conjugation composed with the antipodal map 
    $\varphi \mapsto \varphi + \pi$, which contributes 
    $e^{ik\pi} = (-1)^k$. (Equivalently: the $\delta$-action 
    on the maximal torus $T \subset O(2)$ is the Weyl reflection 
    $\varphi \mapsto -\varphi$, and the Weyl group sign on a 
    weight $k$ is $(-1)^k$.)
  \item The $O(N)$-factor: $\delta$ sends $\lambda$ to its 
    contragredient $\lambda^*$. The intertwiner 
    $V_\lambda \to V_{\lambda^*}$ contributes $(-1)^{\epsilon(\lambda)}$, 
    where $\epsilon(\lambda) = 0$ if the intertwiner is $+1$ 
    (i.e., $V_\lambda$ is self-contragredient with symmetric 
    bilinear form) and $\epsilon(\lambda) = 1$ if $-1$ 
    (antisymmetric form).
\end{enumerate}
The total intertwiner sign is $(-1)^k \cdot (-1)^{\epsilon(\lambda)} 
= (-1)^{k + \epsilon(\lambda)}$.
\end{proof}

\subsection{Uniform $\delta$-Evenness}
\label{subsec:disjointness}

To connect the abstract $\delta$-eigenvalue formula 
(Corollary~\ref{cor:delta-eigenvalue}) to the lattice coordinates 
of Appendix~\ref{app:branching}, we need:

\begin{lemma}[Identification of $\epsilon(\lambda)$ with $|\lambda|$]
\label{lem:epsilon-identification}
In the rank-2 lattice parametrisation $(m_1, m_2)$ of $K$-types 
on the Lie sphere (Appendix~\ref{app:peter-weyl}), the 
contragredient intertwiner sign $\epsilon(\lambda)$ of 
Corollary~\ref{cor:delta-eigenvalue} satisfies:
\begin{equation}
  \epsilon(\lambda) \equiv |\lambda| \equiv m_1 - m_2 \pmod{2}.
\end{equation}
\end{lemma}

\begin{proof}
The involution $\delta = \alpha \circ \theta$ acts on the Lie 
sphere $\check{S} \cong (S^1 \times S^{N-1})/\Z_2$ by exchanging 
the two restricted root directions. We claim that in the 
$(m_1, m_2)$ lattice parametrisation of 
Appendix~\ref{app:peter-weyl}, this action is:
\begin{equation}
\label{eq:delta-swap}
  \delta: V_{(m_1, m_2)} \mapsto V_{(m_2, m_1)}.
\end{equation}

\emph{Proof of \eqref{eq:delta-swap} for general $N$.} The spin 
factor Jordan algebra $V_N = \R \oplus \R^{N-1}$ has rank 2 for 
all $N \geq 3$, with Jordan frame $\{e_1, e_2\}$ given by 
$e_1 = \frac{1}{2}(1, e_N)$ and $e_2 = \frac{1}{2}(1, -e_N)$, 
where $e_N$ is a unit vector in $\R^{N-1}$. Every element 
$u \in V_N$ has a spectral decomposition $u = \lambda_1 e_1 + 
\lambda_2 e_2$ with eigenvalues $\lambda_j = u_0 \pm |u'|$ 
(where $u = (u_0, u')$ in the $\R \oplus \R^{N-1}$ splitting). 
The cone $\Omega = \{u : \lambda_1, \lambda_2 > 0\}$ is the 
Lorentz cone, and the Shilov boundary is 
$\check{S} = \{u \in V_N : \lambda_1 \lambda_2 = 0, 
\lambda_1^2 + \lambda_2^2 = 1\}$.

The involution $\alpha$ acts on the tube domain $T_\Omega$ by 
$\alpha(z) = -\bar{z}$ (negating the real part, 
cf.~Definition~\ref{def:involutions}). On the boundary 
($y = 0$), this sends $u \mapsto -u$, hence 
$(\lambda_1, \lambda_2) \mapsto (-\lambda_1, -\lambda_2)$. 
The Cartan involution $\theta$ acts on $\g = \fk \oplus \fp$ 
as $+1$ on $\fk$ and $-1$ on $\fp$. Its effect on the boundary 
is: $\theta$ fixes the maximal compact subalgebra but negates 
the noncompact directions, which in the Jordan-algebraic picture 
corresponds to $(\lambda_1, \lambda_2) \mapsto 
(\lambda_2, \lambda_1)$ (interchanging the two spectral values). 
To verify this last claim: the maximal abelian subspace 
$\fa \subset \fp$ is one-dimensional (rank of $G/K$ is 2 as a 
Hermitian symmetric space, but the restricted rank is 2 for the 
boundary action), parametrised by $\mathrm{diag}(a_1, a_2)$ in 
the spectral decomposition. The restricted Weyl group 
$W(\g, \fa)$ for the $B_2$ root system has order~8, generated 
by two simple reflections $s_\alpha$ (short root) and 
$s_\beta$ (long root). The element that permutes the spectral 
values $(a_1, a_2) \mapsto (a_2, a_1)$ is the long root 
reflection $s_\beta$, and $\delta$ implements precisely this 
element. (For $N = 4$, the restricted root system degenerates 
to $A_1 \times A_1$, and the spectral permutation is 
the exchange of the two $A_1$ factors.)

Combining: $\delta = \alpha \circ \theta$ sends 
$(\lambda_1, \lambda_2) \mapsto (-\lambda_2, -\lambda_1)$. On 
the $K$-type lattice, where $(m_1, m_2)$ are the exponents of 
the characters $\chi_{m_1} \bar{\chi}_{m_2}$ in the 
Peter--Weyl decomposition, the action of $\delta$ sends 
$(m_1, m_2) \mapsto (m_2, m_1)$ (swapping the two lattice 
indices). This can also be verified directly from the 
$N = 6$ case (Section~\ref{sec:n4}), where it reduces to the 
standard exchange of the two $\SU(2)$ factors in 
$\SO(6) \cong \SU(4)$.

Now we analyse the consequences for the $\delta$-intertwiner 
sign. For $m_1 \neq m_2$, the $K$-types $V_{(m_1, m_2)}$ and 
$V_{(m_2, m_1)}$ are distinct irreducible $K$-modules, and 
$\delta$ exchanges them. These $K$-types are \emph{not} 
individually $\delta$-stable; $\delta$ is an outer involution 
that permutes pairs. The $\delta$-intertwiner 
$V_{(m_1, m_2)} \to V_{(m_2, m_1)}$ has sign $(-1)^{m_1 - m_2} 
= (-1)^{|\lambda|}$, as determined by the action on highest 
weight vectors: the representation-theoretic intertwiner 
$V_{(m_1,m_2)} \to V_{(m_2,m_1)}$ acquires a sign 
$(-1)^{m_1 - m_2}$ from the action of the outer automorphism 
on the highest weight space 
(cf.\ \cite{Knapp2002}, Ch.~VII, \S7, for the general theory 
of intertwiner signs on $K$-types under outer involutions). 
Note that this sign is a property of the representation-theoretic 
intertwiner, not of a sequence of Weyl reflections: in the $B_2$ 
restricted root system, the transposition $(m_1, m_2) \mapsto 
(m_2, m_1)$ corresponds to a \emph{single} simple reflection 
(the short root reflection), and the sign $(-1)^{|m_1 - m_2|}$ 
arises from the induced action on the highest weight vector of 
the $K$-module.

For $m_1 = m_2$, the $K$-type $V_{(m_1, m_1)}$ is $\delta$-stable 
(since $\delta$ fixes the pair $(m_1, m_1)$). By Schur's lemma, 
$\delta$ acts as $\pm 1$ on this irreducible $K$-module. The sign 
is $+1$ for the following reason: the $K$-types with $m_1 = m_2$ 
correspond (via $k = 2m_1$, $\lambda = 0$) to the spherical 
representations---those containing a $K$-fixed vector. By the 
Cartan--Helgason theorem (\cite{Helgason2000}, Ch.~V, \S4), 
spherical representations of $(G, K)$ are precisely those whose 
highest weight is invariant under the restricted Weyl group. 
Since $\delta$ implements a Weyl group element that fixes the 
diagonal $m_1 = m_2$, the $\delta$-intertwiner on spherical 
representations is the identity. (Alternatively: when 
$\lambda = 0$, the $O(N)$-representation is trivial, 
so $\epsilon(\lambda) = 0$ and the eigenvalue is 
$(-1)^{k + 0} = (-1)^{2m_1} = +1$.)

Equating with the abstract formula from 
Corollary~\ref{cor:delta-eigenvalue}: the $\delta$-intertwiner 
sign is $(-1)^{k + \epsilon(\lambda)}$, and we have shown it 
equals $(-1)^{|\lambda|}$. Therefore:
\begin{equation}
  (-1)^{k + \epsilon(\lambda)} = (-1)^{|\lambda|},
\end{equation}
giving $k + \epsilon(\lambda) \equiv |\lambda| \pmod{2}$, i.e., 
$\epsilon(\lambda) \equiv |\lambda| + k \equiv 2m_1 \pmod{2}$.

The conclusion is: $k + \epsilon(\lambda) \equiv 2m_1 \pmod{2}$, 
which is always even since $m_1 \in \Z$.
\end{proof}

\begin{theorem}[Main technical result]
\label{thm:disjointness}
For all $N \geq 3$ and all $K$-types $V_{(k, \lambda)}$ appearing 
in the holomorphic discrete series $\pi_{N/2}$ of $\SO_0(2,N)$:
\begin{equation}
  (-1)^{k + \epsilon(\lambda)} = +1.
\end{equation}
That is, every $K$-type in the Hardy space is $\delta$-even.
\end{theorem}

\begin{proof}
The proof follows immediately from the rank-2 lattice structure 
of $K$-types on the Lie sphere. By the Peter--Weyl decomposition 
(Appendix~\ref{app:peter-weyl}), $K$-types on $\check{S}$ are 
indexed by pairs $(m_1, m_2) \in \Z^2$ with $m_1 \geq m_2$. 
The holomorphic discrete series selects the semigroup 
$m_1 \geq m_2 \geq 0$ (Appendix~\ref{app:hardy-selection}).

In the $(k, |\lambda|)$ coordinates of the body text, where 
$k = m_1 + m_2$ and $|\lambda| = m_1 - m_2$:
\begin{equation}
  k + \epsilon(\lambda) = k + |\lambda| = 2m_1 \equiv 0 \pmod{2}
\end{equation}
for every $(m_1, m_2) \in \Z^2$, since $m_1$ is an integer. 
Hence $(-1)^{k + \epsilon(\lambda)} = +1$ for every $K$-type 
in $H^2(\check{S})$.

The parity constraint is not a consequence of the Hardy condition 
$m_2 \geq 0$; it holds for \emph{all} $K$-types in $L^2(\check{S})$. 
The Hardy condition is needed to ensure holomorphic extension, 
not to enforce parity. See Appendix~\ref{app:parity} for the 
implications in the vector-valued case.
\end{proof}

\begin{remark}[Reinterpretation of the algebraic obstruction]
\label{rem:why-algebra-fails}
This theorem reveals that the $\delta$-odd obstruction identified 
in \cite{Bridge} is more nuanced than originally understood. In 
the \emph{scalar} sector, $\delta$-odd $K$-types do not exist on 
the Lie sphere at all: the parity $k + |\lambda| = 2m_1$ is even 
for \emph{every} $(m_1, m_2)$ in the Peter--Weyl decomposition 
of $L^2(\check{S})$, not merely for those in the Hardy subspace 
$H^2(\check{S})$ (which additionally requires $m_2 \geq 0$). 
The algebraic obstruction therefore cannot arise from 
scalar boundary modes.

To clarify a natural question: if all scalar $K$-types are 
$\delta$-even, why did \cite{Bridge} find $\delta$-odd 
obstructions? The answer is that the algebraic transfer in 
\cite{Bridge} operates on the \emph{full Hilbert space} of the 
theory, which includes vector-valued sections (i.e., fields 
carrying nontrivial fiber representations). The obstructions 
found in \cite{Bridge} arise from the fiber intertwiner $J_\tau$ 
acting on these vector-valued $K$-types, \emph{not} from the 
scalar boundary $K$-type lattice. The present paper's 
contribution is to show that the scalar lattice parity 
($k + |\lambda| = 2m_1$ always even) resolves the scalar 
component of the obstruction, reducing the entire problem to 
the single fiber sign $J_\tau$.

The obstruction lives entirely in the \emph{fiber representation} 
of vector-valued fields: it is controlled by the intertwiner 
$J_\tau$ (Definition~\ref{def:delta-structure}). For fields with 
$J_\tau = +\mathrm{id}$ (bosonic fields, Dirac fermions in even 
dimensions), the combined $\delta$-eigenvalue on Hardy space 
sections is $(-1)^{2m_1} \cdot J_\tau = +1$, and the transfer 
preserves positivity.

The Szeg\H{o} projection plays a different role than initially 
anticipated: it does not ``filter out $\delta$-odd modes'' (which 
are absent from the scalar lattice) but rather projects onto the 
Hardy space $H^2(\check{S})$, ensuring that only positive-energy, 
holomorphically extendable boundary data participates in the 
transfer. The non-Hardy modes it discards (those with $m_2 < 0$) 
are $\delta$-\emph{even} but non-physical: they violate the 
spectrum condition. The algebraic approach fails because it 
operates on all of $L^2(\check{S})$---including these non-Hardy 
modes---without the holomorphic constraint that the analytic 
approach enforces automatically.
\end{remark}


%=============================================================================
\section{The Main Theorem}
\label{sec:main-theorem}
%=============================================================================

We can now prove the main result, resolving Conjecture~\ref{conj:main} 
for fields with positive $\delta$-structure.

\begin{theorem}[Simultaneous reflection positivity via the Cauchy--Szeg\H{o} kernel]
\label{thm:main}
Let $\{\mathcal{W}_n^{(\sigma)}\}_{n \geq 2}$ be the Wightman 
distributions of a reflection-positive quantum field theory on a 
real form $\R^{p,q}$ of $\C^d$ (with $p + q = d$), with fields 
valued in a $\delta$-stable homogeneous vector bundle with 
fiber intertwiner $J_\tau = +\mathrm{id}$ 
(Definition~\ref{def:delta-structure}; this includes all bosonic 
fields and Dirac fermions in even spacetime dimensions). 

By the Bargmann--Hall--Wightman theorem, each $\mathcal{W}_n^{(\sigma)}$ 
is the distributional boundary value of a holomorphic function 
$W_n(z_1, \ldots, z_n)$ defined in the permuted extended tube 
$\mathcal{T}_n \subset (\C^d)^n$. By 
Lemma~\ref{lem:split-regularity} below, the BHW holomorphic 
continuation $W_n$ admits tempered distributional boundary values 
on every standard real form of $\C^d$, including the split real 
form $\R^{d/2,d/2}$. Define the \emph{Szeg\H{o} 
transfer} to any other real form $\R^{p',q'}$ as the 
distributional boundary value of $W_n$ along the target real form:
\begin{equation}
\label{eq:szego-transfer-distributional}
  \mathcal{W}_n^{(\sigma')}(\varphi_1, \ldots, \varphi_n) = 
  \lim_{y \to 0^+} \int W_n(x_1' + iy_1, \ldots, x_n' + iy_n)\,
  \prod_j \varphi_j(x_j')\, dx_j',
\end{equation}
where $\varphi_j \in \mathscr{S}(\R^N)$ are test functions and 
the limit is taken with $y_j \in \Omega$ approaching the 
target real boundary along the appropriate totally real 
submanifold (Proposition~\ref{prop:spacetime-domain}): for the 
Lorentzian target, $y_j \to 0$ directly; for the Euclidean 
target, the limit is the restriction to the imaginary axis 
$i\Omega$; for the split target, $y_j \to 0$ along the 
split-signature causal cone. The limit exists in the distributional 
sense by standard edge-of-the-wedge estimates 
\cite{StreaterWightman} and Hardy space boundary value results 
\cite{SteinWeiss1971}.

When boundary values are sufficiently regular (e.g., for free 
or perturbative theories), this transfer admits the kernel 
representation:
\begin{equation}
\label{eq:szego-transfer}
  \mathcal{W}_n^{(\sigma')}(x_1', \ldots, x_n') = 
  \int_{\check{S}^n} \prod_{j=1}^n S(z_j'(\sigma'), \zeta_j)\,
  \mathcal{W}_n^{(\sigma)}(\zeta_1, \ldots, \zeta_n)\,
  \prod_{j=1}^n d\sigma(\zeta_j),
\end{equation}
where $z_j'(\sigma')$ denotes the embedding of the point $x_j'$ 
on the target real form into the complexified domain, and $S$ is 
the Cauchy--Szeg\H{o} kernel of the Type~IV domain 
$D^N_{IV}$ with $N = 2d - 2$.

\begin{remark}[Analytic framework]
\label{rem:distributional}
The defining equation \eqref{eq:szego-transfer-distributional} 
requires no Szeg\H{o} projection of test functions: the 
holomorphic function $W_n$ already exists in the complex interior 
(by BHW), and the transfer is simply the boundary value of this 
function on a different real form. The Szeg\H{o} kernel enters 
in the kernel representation \eqref{eq:szego-transfer} as the 
mediating kernel between boundary components of the tube domain, 
but the \emph{existence} of the transfer is guaranteed by BHW 
plus edge-of-the-wedge, independently of the kernel formula. The 
role of the Szeg\H{o} kernel in this paper is therefore primarily 
\emph{structural}: it provides the explicit formula and, more 
importantly, the representation-theoretic analysis 
(Theorem~\ref{thm:disjointness}) that explains why the 
$\delta$-odd obstruction does not arise.
\end{remark}

\begin{lemma}[Split-signature boundary regularity]
\label{lem:split-regularity}
Let $\{\mathcal{W}_n^{(\sigma)}\}$ be the Wightman distributions of 
a quantum field theory satisfying the standard Wightman axioms on a 
real form $\R^{p,q}$ of $\C^d$. Then the BHW holomorphic continuation 
$W_n$ admits tempered distributional boundary values on the split 
real form $\R^{d/2,d/2}$ (and on every other standard real form).
\end{lemma}

\begin{proof}
The argument proceeds in three steps.

\emph{Step 1.} The Wightman axioms guarantee that each $\mathcal{W}_n$ 
is a tempered distribution. By the Bargmann--Hall--Wightman theorem 
\cite{StreaterWightman}, $\mathcal{W}_n$ extends to a holomorphic 
function $W_n$ on the permuted extended tube $\mathcal{T}'_n \subset 
(\C^d)^n$, with polynomial growth bounds inherited from the 
temperedness of the source distributions.

\emph{Step 2.} By \cite[Lemma~3.4]{SplitWedge}, the split tube 
$T_S = \R^d + iV^+_S$ is contained in the permuted extended tube: 
$T_S \subset \mathcal{T}'_n$ (for the single-variable case $n = 2$; 
the multi-variable case follows by taking products of forward tubes 
in the difference variables). Therefore $W_n$ restricts to a 
holomorphic function on $T_S$ with polynomial growth.

\emph{Step 3.} By the theorem of Vladimirov \cite{Vladimirov1966} 
on distributional boundary values of holomorphic functions in tube 
domains: a holomorphic function in a tube $T_\Gamma = \R^k + i\Gamma$ 
with polynomial growth admits tempered distributional boundary values 
as $\mathrm{Im}(z) \to 0$ within $\Gamma$. Applying this with 
$\Gamma = V^+_S$ (the split forward cone, which is an open convex 
cone by \cite[Lemma~3.4]{SplitWedge}), we obtain that $W_n$ has 
tempered distributional boundary values on $\R^{d/2,d/2}$.

The same argument applies to every standard real form: the Lorentzian 
boundary values are the original Wightman distributions (by 
definition), the Euclidean boundary values are the Schwinger 
functions (by OS reconstruction), and the split boundary values 
exist by the above argument.
\end{proof}

\begin{remark}[$n$-point singularity structure]
\label{rem:npoint-regularity}
For $n > 2$, the Wightman distributions have singularities on 
coincident-point configurations. The Szeg\H{o} kernel 
$S(z, \zeta)$ is singular on the boundary ($\nu = N/2$ lies near 
the lower end of the continuous Wallach set, per Remark~\ref{rem:wallach}), 
and the product $\prod_{j=1}^n S(z_j', \zeta_j)$ in 
\eqref{eq:szego-transfer} amplifies these singularities.
However, the tube domain framework sidesteps this difficulty: the 
holomorphic function $W_n$ is non-singular in the interior 
$\mathcal{T}_n$, and its boundary values on the target real form 
inherit the same distributional regularity as the source Wightman 
distributions by Lemma~\ref{lem:split-regularity}.
\end{remark}

Then:
\begin{enumerate}[label=(\roman*)]
  \item $\mathcal{W}_n^{(\sigma')}$ satisfies the Osterwalder--Schrader 
    axioms with respect to the natural reflection positivity involution 
    of the target signature $(p', q')$.
  \item The transfer \eqref{eq:szego-transfer} is independent of the 
    choice of intermediate complexification, depending only on the 
    source and target real forms.
  \item The transfer preserves the spectrum condition: if the original 
    theory has positive energy, so does the transferred theory.
  \item The transfer preserves clustering: if the original theory 
    has a unique vacuum, so does the transferred theory.
  \item The transfer is \emph{covariant}: if the original theory 
    transforms under a representation of $\SO_0(p,q)$, the 
    transferred theory transforms under the corresponding 
    representation of $\SO_0(p',q')$, with the intertwining 
    mediated by the analytic continuation through $\SO(d, \C)$.
\end{enumerate}
\end{theorem}

\begin{proof}
\textbf{(i) Reflection positivity.} We give the proof in three 
stages: first the 2-point case (which is the heart of the 
argument), then the extension to $n$-point functions.

\emph{Stage 1: Euclidean positivity.} The source theory satisfies 
OS positivity, which (by the Osterwalder--Schrader reconstruction 
theorem) is equivalent to the positivity of the Schwinger 
functions $S_n^{(\sigma)}$ on the Euclidean section. The 
Euclidean section embeds into the tube domain $T_\Omega$ as the 
imaginary axis $i\Omega$ (Proposition~\ref{prop:spacetime-domain}). 
On this section, the Cauchy--Szeg\H{o} kernel restricts to 
$S(iy, iy') = \Delta(y + y')^{-N/2}$, which is positive-definite 
by Theorem~\ref{thm:scalar-positivity}.

\emph{Stage 2: 2-point transfer.} The OS positivity condition 
for the 2-point function is: for all test functions $f$ supported 
in the positive-time half-space,
\begin{equation}
\label{eq:os-positivity}
  \sum_{i,j} \bar{c}_i c_j\, \mathcal{W}_2^{(\sigma)}(\theta x_i, x_j) 
  \geq 0.
\end{equation}
By the BHW theorem, $\mathcal{W}_2^{(\sigma)}$ is the boundary 
value of a holomorphic function $W_2(z_1, z_2)$ in the tube. 
The transferred 2-point function $\mathcal{W}_2^{(\sigma')}$ is 
the boundary value of the \emph{same} $W_2$ on the target real 
form. We must show \eqref{eq:os-positivity} holds with $\sigma$ 
replaced by $\sigma'$ and $\theta$ by $\theta'$.

The key is that both the source and target OS conditions can be 
\emph{analytically continued to the Euclidean section}, where 
they become the same condition: positivity of 
$\Delta(y + y')^{-N/2}$ (Theorem~\ref{thm:scalar-positivity}). 
The only difference between the source and target is the 
\emph{$\delta$-eigenvalue} that arises when the Euclidean 
reflection $\theta_E$ is analytically continued to the target 
reflection $\theta'$. By Theorem~\ref{thm:disjointness}, all 
scalar $K$-types are $\delta$-even, so this eigenvalue is $+1$. 
By Lemma~\ref{lem:vv-delta-eigenvalue} and the hypothesis 
$J_\tau = +\mathrm{id}$, the vector-valued $\delta$-eigenvalue 
is also $+1$. Therefore the positivity condition 
\eqref{eq:os-positivity} holds identically on the target.

\emph{Stage 3: $n$-point functions.} Reflection positivity is 
a condition on the \emph{full family} of Schwinger functions 
$\{S_n\}$, not merely on $S_2$. The Osterwalder--Schrader axiom 
(OS0) requires that for any finite sequence 
$F = (f_0, f_1, \ldots, f_M)$ of test functions with each $f_n$ 
supported at positive Euclidean time, the sesquilinear form
\begin{equation}
\label{eq:os-general}
  \langle F, F \rangle_{\mathrm{OS}} := 
  \sum_{m,n=0}^{M} S_{m+n}\!\left(\theta_E f_m \otimes f_n\right) 
  \geq 0
\end{equation}
is nonnegative. The OS reconstruction theorem then builds a 
pre-Hilbert space $\mathcal{H}$ from this form by quotienting 
out the null space.

We establish positivity of the transferred OS form in three steps.

\emph{Step 3a: Variable-by-variable structure.} By translation 
invariance, each Wightman function $W_n$ depends on $n-1$ difference 
variables $\xi_j = x_{j+1} - x_j$, each living in $\C^d$. The 
(un-permuted) forward tube $T_n = \{(\xi_1, \ldots, \xi_{n-1}) : 
\mathrm{Im}(\xi_j) \in V^+ \text{ for all } j\}$ is a product 
of single-variable tubes. On this product domain, the Szeg\H{o} 
transfer acts on each difference variable independently: the 
transfer kernel factorises as 
$\prod_{j=1}^{n-1} S(z'_j(\sigma'), \zeta_j)$, so the $n$-point 
transfer is an $(n-1)$-fold tensor product of the 2-point transfer 
established in Stage~2.

\emph{Step 3b: Extension to the permuted extended tube.} The full 
$n$-point Wightman function lives on the permuted extended tube 
$\mathcal{T}'_n$, which is larger than the forward tube $T_n$ 
and is not a product. However, $\mathcal{T}'_n$ is obtained from 
$T_n$ by the action of the permutation group composed with 
$\SO(d, \C)$ transformations (the Jost--Lehmann--Dyson 
representation). The Cauchy--Szeg\H{o} kernel is $G$-covariant: 
$S(g \cdot z, g \cdot \zeta) = S(z, \zeta)$ for 
$g \in G = \SO_0(2, N)$, which contains $\SO(d, \C)$ via the 
complexified isometry group. Therefore the transferred $n$-point 
function, initially defined on the product of single-variable 
tubes by Step~3a, extends to $\mathcal{T}'_n$ by $\SO(d, \C)$ 
covariance, and the extension satisfies the same permutation 
symmetry at Jost points as the source theory.

\emph{Step 3c: Positivity of the full OS form.} Since the 
$n$-point transfer is a tensor product of 2-point transfers 
(Step~3a), and each 2-point transfer preserves positivity 
(Stage~2), the OS form \eqref{eq:os-general} on the target 
real form is a sum of terms, each of which is a product of 
positive 2-point pairings. Explicitly: the OS form decomposes 
with respect to the $K$-type structure as
\begin{equation}
  \langle F, F \rangle_{\mathrm{OS}}^{(\sigma')} = 
  \sum_{m,n} S_{m+n}^{(\sigma')}(\theta' \bar{f}_m \otimes f_n),
\end{equation}
where each $S_{m+n}^{(\sigma')}$ is the boundary value of the 
\emph{same} holomorphic function $W_{m+n}$ as the source theory 
(only the boundary changes). The $\delta$-eigenvalue controlling 
the sign under $\theta_E \to \theta'$ is $+1$ on every $K$-type 
(Theorem~\ref{thm:disjointness} + $J_\tau = +\mathrm{id}$). 
Furthermore, the $K$-action commutes with the OS involution 
$\theta'$ (since $\theta'$ is implemented by elements of the 
normaliser of $K$ in $G$, being a composition of the geometric 
reflection and the $R$-symmetry from \cite{SplitWedge}), so the 
OS form is block-diagonal with respect to $K$-types. 
Positivity of each block (established by the tensor product 
structure of Step~3a and the 2-point positivity of Stage~2) 
gives positivity of the full form.

\textbf{(ii) Independence.} The Cauchy--Szeg\H{o} kernel is the 
unique reproducing kernel for the Hardy space $H^2(D^N_{IV})$. Since 
the Hardy space is determined by the domain (which depends only on 
$d$, not on the choice of real form), and since the embedding of each 
real form into the complexification is canonical up to the $V_4$ 
action, the transfer is canonical.

\textbf{(iii) Spectrum condition.} In the tube realisation, the 
Cauchy--Szeg\H{o} kernel \eqref{eq:szego-tube} is the Fourier--Laplace 
transform of the Riesz measure supported on the forward light cone 
$\Omega$ (equation \eqref{eq:fourier-laplace}). The Szeg\H{o} 
projection therefore maps $L^2(\R^N)$ to functions whose Fourier 
support lies in $\overline{\Omega}$---this is exactly the 
positive-energy condition.

The ``spectrum condition'' takes different forms on different real 
forms: on the Lorentzian form $\R^{1,N-1}$, it is support of the 
Fourier transform in the closed forward light cone; on the 
Euclidean form $\R^N$, it corresponds to the OS growth condition 
(exponential decay in the Euclidean time direction); on the split 
form, it is support in the corresponding causal cone. These 
conditions are unified by the tube domain picture: the holomorphic 
function $W_n$ in the permuted extended tube has Fourier--Laplace 
support in the \emph{dual cone} $\Omega^* = \Omega$ (self-duality 
of the Lorentz cone). When the boundary value is taken along 
different real forms, the dual cone projects to the relevant 
spectrum condition for that signature. The transfer preserves 
spectral support because it is mediated by a holomorphic function 
in the common tube domain, whose Fourier--Laplace support is 
invariant under boundary restriction.

\textbf{(iv) Clustering.} The cluster decomposition property 
is equivalent to the uniqueness of the vacuum vector in the 
GNS Hilbert space, which is the lowest $K$-type $V_{(0, \mathrm{triv})}$. 
Since this is a one-dimensional $\delta$-even subspace, it is preserved 
by the transfer map (Theorem~\ref{thm:disjointness}, base case). 
Uniqueness on the source side implies uniqueness on the target side.

\textbf{(v) Covariance.} The Cauchy--Szeg\H{o} kernel is 
$G$-invariant: $S(g \cdot z, g \cdot \zeta) = S(z, \zeta)$ for 
$g \in G = \SO_0(2, N)$. The real form subgroups 
$\SO_0(p,q) \hookrightarrow G$ act on the corresponding boundary 
components, and the kernel intertwines these actions. The 
transferred $n$-point functions therefore inherit the covariance 
properties of the source theory, with the symmetry group replaced 
by the isometry group of the target real form.
\end{proof}


%=============================================================================
\section{Discussion}
\label{sec:discussion}
%=============================================================================

\subsection{What the Theorem Says}

We have established that all scalar $K$-type labels on the Lie
sphere are $\delta$-even ($k + |\lambda| = 2m_1$, always even),
and that the Cauchy--Szeg\H{o} kernel provides a holomorphic
filter selecting precisely the Hardy-space $K$-types that extend
into the tube domain. The three formulations (Euclidean,
Lorentzian, split) are connected by continuation through the
interior of the Type~IV domain, with the Szeg\H{o} projection
providing the explicit integral transform. The $K$-type label
compatibility proved here is a necessary algebraic ingredient
for simultaneous reflection positivity; the operator-level
reconstruction additionally requires the antiunitary/standard
subspace framework of \cite{NO17,MN21} and the covering group
classification of \cite{NeebPIM} (see \S\ref{subsec:covering-groups}).

\subsection{Why Analysis Succeeds Where Algebra Fails}

The algebraic transfer of \cite{Bridge} attempts to map states 
\emph{directly} between real forms, operating on the full $L^2$ 
space of boundary data. It encounters sign reversals in the inner 
product that destroy reflection positivity.

The analytic transfer threads through the complex interior. The 
Szeg\H{o} projection enforces holomorphicity, which automatically 
restricts attention to the Hardy space $H^2(\check{S})$---the 
positive-energy subspace of $L^2$. The non-Hardy modes that the 
algebraic approach attempts (and fails) to transfer are not 
$\delta$-odd obstructions; they are non-physical modes that 
fail to extend holomorphically into the tube domain, which by 
the Paley--Wiener theorem for tube domains 
\cite{SteinWeiss1971} is equivalent to violating the 
spectrum condition (Fourier support outside the closed forward 
cone $\overline{\Omega}$).

A key lesson of the present analysis is that all scalar $K$-type
labels are $\delta$-even: the identity $k + |\lambda| = 2m_1$
holds in the full $K$-type lattice, not merely in $H^2(\check{S})$
but in $L^2(\check{S})$ as well (Theorem~\ref{thm:disjointness},
Appendix~\ref{app:parity}). In the vector-valued sector,
the $\delta$-eigenvalue on labels is controlled entirely by the fiber
intertwiner $J_\tau$ (Lemma~\ref{lem:vv-delta-eigenvalue}), and
for physical bosonic fields ($J_\tau = +\mathrm{id}$) the Hardy
space is uniformly $\delta$-even at the label level. (See
Remark~\ref{rem:distributional} for the respective roles of the
Cauchy--Szeg\H{o} kernel and the Bargmann--Hall--Wightman theorem
in this result.)

As noted in Remark~\ref{rem:label-vs-operator}, the $K$-type label
preservation proved here is necessary but not sufficient for the
full operator-level reconstruction. In positive-energy theories,
time reflection is antiunitary \cite{NO17}, and the operator
framework involves standard subspaces and the BGL construction on
covering groups \cite{MN21}, with the relevant covering depending
on the spacetime dimension mod~4 \cite[\S5.5]{Bridge}.
The label preservation proved here ensures compatibility of the
$\delta$-action with the $K$-type decomposition, while the
modular-theoretic machinery provides the operator-level positivity.

\begin{remark}[Relation to the Bisognano--Wichmann property]
\label{rem:hardy-vs-bw}
Conjecture~\ref{conj:main} (from \cite{Bridge}) was originally 
formulated using the Bisognano--Wichmann (BW) property---the 
statement that the modular operator of the wedge algebra is 
the boost generator. The present paper replaces this with Hardy 
space membership: boundary data belong to $H^2(\check{S})$ if 
and only if their Fourier--Laplace transform is supported in the 
closed forward cone $\overline{\Omega}$ (Paley--Wiener theorem 
for tube domains \cite{SteinWeiss1971}).

These conditions are closely related. The BW property implies 
that vacuum expectation values extend holomorphically into a 
strip of width $\pi$ in the boost parameter (via the KMS 
condition). In the presence of the full Wightman axioms---especially 
the spectrum condition and Lorentz covariance---this local 
extension bootstraps to the global Hardy space extension via 
the BHW theorem. Conversely, Hardy space membership (global 
holomorphic extension to the full tube) implies the BW strip 
analyticity by restriction.

Thus, for theories satisfying the Wightman axioms, Hardy space 
membership and the BW property are equivalent. The Hardy space 
formulation is the more fundamental one for our purposes: it is 
an intrinsic condition on the Szeg\H{o} projection, expressed 
directly in terms of the Type~IV domain geometry, whereas the 
BW property requires the full Tomita--Takesaki modular theory 
apparatus. Theorem~\ref{thm:main} therefore implies 
Conjecture~\ref{conj:main} of \cite{Bridge} (not merely a 
variant), with the Hardy space condition providing the precise 
mechanism that the BW property leaves implicit.
\end{remark}

\subsection{Scope and Limitations}

Theorem~\ref{thm:main} applies to $\delta$-stable bundles with 
$J_\tau = +\mathrm{id}$. This includes all bosonic/tensor fields 
and Dirac fermions in even spacetime dimensions 
(Remark~\ref{rem:physical-reps}). Two cases remain:

\begin{itemize}
  \item \textbf{Odd-dimensional spinors} ($J_\tau = -\mathrm{id}$): 
    The Hardy space is uniformly $\delta$-odd 
    (Proposition~\ref{prop:vv-filter}(ii)). The Cauchy--Szeg\H{o} 
    kernel itself is unchanged, but reflection positivity must be 
    formulated with respect to the \emph{graded} inner product 
    standard in the fermionic Osterwalder--Schrader framework.
    
    Specifically, fermionic OS positivity replaces the bosonic 
    condition $\langle \theta f, f \rangle \geq 0$ with 
    $\langle \theta f, \Gamma f \rangle \geq 0$, where $\Gamma$ 
    is the grading operator (here $\Gamma = J_\tau$ on the fiber). 
    This is the standard modification for spinor fields and accounts 
    for the sign flip under $2\pi$-rotation 
    (Remark~\ref{rem:Jtau-physics}). The scalar kernel positivity 
    (Theorem~\ref{thm:scalar-positivity}) is unaffected by this 
    modification: it is the \emph{inner product pairing}, not the 
    kernel, that acquires the grading twist. Since the Szeg\H{o} 
    kernel remains positive-definite on the Euclidean section 
    regardless of $J_\tau$, the graded OS axioms should follow from 
    the same analytic machinery. A detailed verification is deferred 
    to a follow-up paper.
  \item \textbf{Non-$\delta$-stable bundles}: These arise for 
    chiral (Weyl) spinors in even dimensions, where the half-spin 
    representations $S^\pm$ are exchanged by $\delta$. The appropriate 
    treatment uses the Dirac bundle $S^+ \oplus S^-$, which is 
    $\delta$-stable with $J_\tau = +\mathrm{id}$. Individual chiral 
    sectors cannot be transferred independently, consistent with the 
    physics (chiral theories are not reflection-positive).
\end{itemize}

A third limitation is \emph{structural}: the $K$-type label
preservation proved here ($k + |\lambda| = 2m_1$) does not by
itself determine the operator implementing time reflection within
each $K$-type (Remark~\ref{rem:label-vs-operator}). In
positive-energy theories, time reflection is antiunitary
\cite{NO17}, and the operator-level analysis requires the standard
subspace formalism and the BGL construction on covering groups
\cite{MN21}, with the covering determined by $d \bmod 4$
(\S\ref{subsec:covering-groups}). The $K$-type label compatibility
proved here is necessary for this construction but not sufficient.

These limitations are \emph{technical}, not fundamental: the
Cauchy--Szeg\H{o} kernel and the Type~IV domain geometry are
universal. Only the inner product formulation and the covering
group structure vary with the field content and spacetime
dimension.

\begin{remark}[Unconditional boundary regularity]
\label{rem:regularity-upgrade}
A notable feature of the present approach is that the boundary 
regularity of the transferred distributions is \emph{proven} 
(Lemma~\ref{lem:split-regularity}), not assumed. The key 
ingredients are: (i) the polynomial growth of the BHW 
holomorphic continuation (from temperedness of Wightman 
distributions), (ii) the split tube inclusion 
$T_S \subset \mathcal{T}'$ from \cite[Lemma~3.4]{SplitWedge}, 
and (iii) Vladimirov's theorem on distributional boundary values 
in tube domains \cite{Vladimirov1966}. This upgrades Theorem~\ref{thm:main} 
from a conditional statement (as it appeared in an earlier version 
of this paper) to an unconditional one for all Wightman theories 
with positive $\delta$-structure.
\end{remark}

\subsection{Covering Groups and Dimension Mod~4}
\label{subsec:covering-groups}

The L\"uscher--Mack theorem produces representations of the
\emph{simply connected} $c$-dual group $G^c$---the universal
covering $\widetilde{\SO}_0(2,N)$, which has infinite centre
$Z \cong \Z$ for $N \geq 3$ \cite[Theorem~6.8]{NO18}. Whether the
resulting BGL net \cite{BGL02,MN21} is local, twisted-local, or
lives on a finite covering depends on the kernel of the
representation. For the scalar Wallach point at
$\nu = d/2 - 1$ of $\SO_0(2,d)$, the classification by
\cite[Theorem~5.35]{NeebPIM} is:
\begin{center}
\begin{tabular}{@{}ll@{}}
\toprule
\textbf{Dimension} & \textbf{Covering} \\
\midrule
$d - 2 \in 4\Z$ & Adjoint group (minimal covering) \\
$d \in 4\Z$ & $\SO_0(2,d)$ with $U(-1) = -1$ (twisted-local) \\
$d$ odd & $2$-fold covering \\
\bottomrule
\end{tabular}
\end{center}
The $K$-type label preservation proved in Theorem~\ref{thm:disjointness}
($k + |\lambda| = 2m_1$ for all $K$-types) is a universal
arithmetic fact, independent of $d \bmod 4$. It establishes that
the $\delta$-action is well-defined on the $K$-type lattice for
\emph{all} dimensions. The covering group classification then
determines the operator-level behaviour, with the general
vector-valued case to be addressed in \cite{MN25}.

\subsection{Relation to Previous Wick Rotation Results}

The analytic continuation of correlation functions between
Euclidean and Lorentzian signatures has a long history, beginning 
with the Osterwalder--Schrader reconstruction theorem 
\cite{OS1973, OS1975}. The L\"uscher--Mack theorem establishes 
that conformal field theories in Euclidean signature analytically 
continue to Lorentzian signature via the conformal group, using 
the tube domain structure of the conformal compactification. Our 
result differs in three respects: (1) it does not require conformal 
invariance---only the Wightman axioms and the BHW theorem; (2) it 
treats all three real forms (Euclidean, Lorentzian, split) 
simultaneously rather than just the Euclidean--Lorentzian pair; 
and (3) the mediating domain is the Type~IV bounded symmetric 
domain $D^N_{IV}$, which is specific to the $\SO(2,N)$ structure 
rather than the conformal group $\SO(2,d)$. The conformal bootstrap 
approach to Wick rotation (which uses crossing symmetry and OPE 
convergence) is complementary: it provides detailed control over 
individual operator contributions, while our approach gives a 
uniform transfer of the full $n$-point structure.

\subsection{The Trilogy in Perspective}

The three papers form a logical progression:

\begin{enumerate}
  \item \textbf{Algebraic structure} (\cite{SplitWedge}): The
    $V_4$ group and the real-form correspondence exist and are
    well-defined. For $d = 4$, the correspondence is perfect.
  \item \textbf{Algebraic obstruction and antiunitary correction}
    (\cite{Bridge}): For $d > 4$, the algebraic transfer encounters
    sign reversals. The obstruction is precisely characterised. The
    paper also identifies the correct resolution framework: time
    reflection must be antiunitary (not unitary) in positive-energy
    theories, and the BGL net construction on covering groups---with
    the covering determined by $d \bmod 4$---replaces the eigenspace
    decomposition with the standard subspace formalism
    \cite{NO17,MN21}.
  \item \textbf{$K$-type label compatibility} (present paper): The
    Cauchy--Szeg\H{o} kernel provides a holomorphic filter that
    selects Hardy-space $K$-types, and the identity
    $k + |\lambda| = 2m_1$ proves that $\delta$ preserves all
    $K$-type labels. This is a necessary algebraic condition for the
    BGL net construction, complementary to the modular-theoretic
    operator framework.
\end{enumerate}

Together with the modular theory of \cite{NO17,MN21} and the
covering group classification of \cite{NeebPIM,MN25}, these
establish the algebraic and analytic infrastructure for simultaneous
reflection positivity on all real forms of $\C^d$, mediated by
the Cauchy--Szeg\H{o} kernel of the Type~IV bounded symmetric
domain. The full operator-level reconstruction for general
representations awaits the vector-valued covering group
classification of \cite{MN25}.


%=============================================================================
% APPENDIX
%=============================================================================
\appendix

\section{Branching Rules for $(\SO_0(2,N), S(O(2) \times O(N)))$}
\label{app:branching}

We provide the explicit branching rules that complete the proof 
of Theorem~\ref{thm:disjointness}. The key simplification is 
that $D^N_{IV}$ has rank $r = 2$, so all $K$-types are indexed 
by \emph{pairs} of integers.

\subsection{Peter--Weyl Decomposition on the Lie Sphere}
\label{app:peter-weyl}

Let $G = \SO_0(2, N)$, $K = S(O(2) \times O(N))$, and let 
$\check{S} \cong (S^1 \times S^{N-1})/\Z_2$ be the Shilov 
boundary (Lie sphere). The regular $K$-representation on 
$L^2(\check{S})$ is \emph{multiplicity-free} and decomposes as:
\begin{equation}
\label{eq:peter-weyl}
  L^2(\check{S}) \cong \widehat{\bigoplus}_{\substack{
    \mathbf{m} = (m_1, m_2) \in \Z^2 \\ m_1 \geq m_2}} 
  V_{\mathbf{m}},
\end{equation}
where $V_{\mathbf{m}}$ is the irreducible $K$-representation 
labelled by the pair $(m_1, m_2)$. This indexing reflects the 
rank-2 structure: the two integers correspond to the two 
restricted roots of the symmetric space $G/K$.

\begin{remark}
The multiplicity-free property is special to the Shilov boundary 
of a bounded symmetric domain and follows from the Cartan--Helgason 
theorem applied to the compact dual. For Type~IV domains, the 
compact dual is $\SO(2+N)/S(O(2) \times O(N))$, and the 
Peter--Weyl decomposition on its Shilov boundary is a classical 
result in the harmonic analysis of symmetric spaces 
\cite{Hua1963, FaKo1994}.
\end{remark}

\subsection{Translation to $(k, |\lambda|)$ Coordinates}
\label{app:translation}

The pair $(m_1, m_2)$ relates to the physical labels used in 
the body of the paper as follows:
\begin{equation}
\label{eq:coordinate-change}
  k := m_1 + m_2, \qquad |\lambda| := m_1 - m_2,
\end{equation}
where:
\begin{itemize}
  \item $k$ is the $O(2)$-weight (Fourier index along the 
    $S^1$ factor of $\check{S}$);
  \item $|\lambda|$ is the ``depth'' of the $O(N)$-highest weight 
    $\lambda$, measuring the complexity of the $O(N)$-representation.
\end{itemize}

The inverse transformation is:
\begin{equation}
\label{eq:inverse-coordinates}
  m_1 = \frac{k + |\lambda|}{2}, \qquad 
  m_2 = \frac{k - |\lambda|}{2}.
\end{equation}
Note that for $(m_1, m_2) \in \Z^2$, both $k + |\lambda|$ and 
$k - |\lambda|$ are automatically even.

\subsection{The Hardy/Holomorphic Selection Rule}
\label{app:hardy-selection}

The Hardy space $H^2(\check{S})$ corresponds to the holomorphic 
$K$-finite boundary values of the holomorphic discrete series. 
By the main theorem (\emph{Satz}) of Schmid \cite{Schmid1969}, 
the $K$-types appearing in $H^2$ are precisely those in the 
\emph{semigroup part} of the Peter--Weyl decomposition---the 
highest-weight indices satisfying $n_1 \geq n_2 \geq \cdots 
\geq n_r \geq 0$ in Schmid's notation, which for rank $r = 2$ 
reduces to:
\begin{equation}
\label{eq:hardy-selection}
  H^2(\check{S}) = \bigoplus_{\substack{
    (m_1, m_2) \in \Z^2 \\ m_1 \geq m_2 \geq 0}} V_{\mathbf{m}}.
\end{equation}

In the $(k, |\lambda|)$ coordinates, the condition $m_2 \geq 0$ 
translates to:
\begin{equation}
\label{eq:hardy-inequality}
  m_2 = \frac{k - |\lambda|}{2} \geq 0 
  \qquad \Longleftrightarrow \qquad k \geq |\lambda|.
\end{equation}
This is the holomorphic extension condition 
\eqref{eq:ktypes-hardy} used in the body of the paper.

\subsection{The Parity Constraint: Proof of Theorem~\ref{thm:disjointness}}
\label{app:parity}

We can now prove the central claim in one line.

\begin{proof}[Proof of Theorem~\ref{thm:disjointness}]
For any $K$-type $V_{\mathbf{m}}$ in the Hardy space, we have 
$m_1, m_2 \in \Z$ with $m_1 \geq m_2 \geq 0$. The 
$\delta$-eigenvalue is determined by the parity of 
$k + |\lambda|$ (Corollary~\ref{cor:delta-eigenvalue}). 
But:
\begin{equation}
\label{eq:parity-proof}
  k + |\lambda| = (m_1 + m_2) + (m_1 - m_2) = 2m_1,
\end{equation}
which is \emph{even} for every $(m_1, m_2) \in \Z^2$. Therefore 
the $\delta$-eigenvalue is $(-1)^{2m_1} = +1$ for every $K$-type 
in $H^2(\check{S})$.
\end{proof}

\begin{remark}
The proof is essentially arithmetic: the rank-2 lattice indexing 
forces $k + |\lambda|$ to be even regardless of any further 
constraints. The Hardy condition $m_2 \geq 0$ was not needed 
for the parity conclusion---but it \emph{is} needed to ensure 
that the $K$-type belongs to the holomorphic discrete series. 
The $\delta$-odd types (those with $k + |\lambda|$ odd) would 
require half-integer values of $m_1$, which do not occur in 
the $K$-type lattice. So the $\delta$-odd types are not merely 
excluded from $H^2$; they do not appear in $L^2(\check{S})$ at all 
for the scalar case.

For \emph{vector-valued} Hardy spaces $H^2(\check{S}, \VV)$, 
the $\delta$-eigenvalue on $V_{\mathbf{m}} \otimes \VV$ is 
$(-1)^{2m_1} \cdot J_\tau = J_\tau$ by 
Lemma~\ref{lem:vv-delta-eigenvalue}. 
The scalar lattice contribution is always $+1$; the entire 
$\delta$-character is determined by the fiber intertwiner $J_\tau$ 
(Definition~\ref{def:delta-structure}). The precise conditions 
under which different fiber representations contribute positive 
or negative $\delta$-structure are given in 
Proposition~\ref{prop:vv-filter} and Remark~\ref{rem:physical-reps}.
\end{remark}

\subsection{Illustration: Non-Hardy $K$-Types at $N = 10$}
\label{app:counterexample}

To illustrate the filter mechanism, we exhibit $K$-types in 
$L^2(\check{S})$ that are excluded from $H^2(\check{S})$.

For $N = 10$ (corresponding to $d = 6$ spacetime dimensions), 
consider the $K$-type with $(m_1, m_2) = (2, -1)$:
\begin{equation}
  k = m_1 + m_2 = 1, \qquad |\lambda| = m_1 - m_2 = 3.
\end{equation}
Since $m_1 \geq m_2$ ($2 \geq -1$), this $K$-type belongs to 
$L^2(\check{S})$. However, $m_2 = -1 < 0$ violates the Hardy 
condition, so it is \emph{not} in $H^2(\check{S})$. Its 
$\delta$-parity is $(-1)^{2m_1} = (-1)^4 = +1$: $\delta$-even 
despite being non-holomorphic.

This example illustrates that for the \emph{scalar} Hardy space, 
all $K$-types in $L^2(\check{S})$ are automatically $\delta$-even 
(since $2m_1$ is always even). The $\delta$-odd obstruction 
identified in \cite{Bridge} therefore cannot arise from the 
scalar lattice; it arises entirely from the fiber 
$\delta$-structure in the vector-valued case.

Specifically, when $\VV$ carries negative $\delta$-structure 
($J_\tau = -\mathrm{id}$), the combined $\delta$-eigenvalue on 
$V_{\mathbf{m}} \otimes \VV$ becomes $(-1)^{2m_1} \cdot (-1) = -1$, 
making every $K$-type in $L^2(\check{S}, \VV)$ $\delta$-odd. 
These sections are still present in $L^2$ but the Szeg\H{o} 
projection onto $H^2(\check{S}, \VV)$ selects the holomorphic 
components, which---by Proposition~\ref{prop:vv-filter}(ii)---are 
uniformly $\delta$-odd. The resolution for such fields 
(odd-dimensional spinors) requires the twisted reflection 
positivity involution discussed in Remark~\ref{rem:physical-reps}(b).


%=============================================================================
% BIBLIOGRAPHY
%=============================================================================
\begin{thebibliography}{99}

\bibitem{OS1973}
K.~Osterwalder and R.~Schrader,
\emph{Axioms for Euclidean Green's functions},
Comm.\ Math.\ Phys.\ \textbf{31} (1973), 83--112.

\bibitem{OS1975}
K.~Osterwalder and R.~Schrader,
\emph{Axioms for Euclidean Green's functions II},
Comm.\ Math.\ Phys.\ \textbf{42} (1975), 281--305.

\bibitem{Bridge}
A.~Abrahams,
\emph{Bridge Triples and the Klein Four-Group for Real Forms
of $\SO(2n,\C)$},
(2026), Paper~2 in the present series. Preprint available
from the author.

\bibitem{SplitWedge}
A.~Abrahams,
\emph{Split Signature as a Third Axiomatization of
Parity-Invariant Quantum Field Theory},
(2026), Paper~1 in the present series. Preprint available
from the author.

\bibitem{Hua1963}
L.K.~Hua,
\emph{Harmonic Analysis of Functions of Several Complex Variables 
in the Classical Domains},
Translations of Mathematical Monographs, Vol.~6,
American Mathematical Society, 1963.

\bibitem{FaKo1994}
J.~Faraut and A.~Kor\'anyi,
\emph{Analysis on Symmetric Cones},
Oxford Mathematical Monographs, Clarendon Press, 1994.

\bibitem{NeebOlafsson2018}
K.-H.~Neeb and G.~\'Olafsson,
\emph{Reflection Positivity: A Representation Theoretic Perspective},
SpringerBriefs in Mathematical Physics, Vol.~32, Springer, 2018.

\bibitem{KrotzStanton2004}
B.~Kr\"otz and R.J.~Stanton,
\emph{Holomorphic extensions of representations: (I) automorphic functions},
Ann.\ of Math.\ \textbf{159} (2004), 641--724.

\bibitem{RossiVergne1976}
H.~Rossi and M.~Vergne,
\emph{Analytic continuation of the holomorphic discrete series 
of a semi-simple Lie group},
Acta Math.\ \textbf{136} (1976), 1--59.

\bibitem{KoranyiWolf1965}
A.~Kor\'anyi and J.A.~Wolf,
\emph{Realization of Hermitian symmetric spaces as generalized half-planes},
Ann.\ of Math.\ \textbf{81} (1965), 265--288.

\bibitem{Schmid1969}
W.~Schmid,
\emph{Die Randwerte holomorpher Funktionen auf hermitesch symmetrischen R\"aumen},
Invent.\ Math.\ \textbf{9} (1969), 61--80.
(The semigroup characterisation of Hardy space $K$-types is the 
main \emph{Satz}, with the highest-weight indexing defined in 
displayed equation~(2).)

\bibitem{HechtSchmid1983}
H.~Hecht and W.~Schmid,
\emph{Characters, asymptotics and $\fn$-homology of Harish-Chandra modules},
Acta Math.\ \textbf{151} (1983), 49--151.

\bibitem{StreaterWightman}
R.F.~Streater and A.S.~Wightman,
\emph{PCT, Spin and Statistics, and All That},
Princeton University Press, Princeton, NJ, 2000 (corrected edition).

\bibitem{SteinWeiss1971}
E.M.~Stein and G.~Weiss,
\emph{Introduction to Fourier Analysis on Euclidean Spaces},
Princeton University Press, Princeton, NJ, 1971.

\bibitem{Helgason2000}
S.~Helgason,
\emph{Groups and Geometric Analysis: Integral Geometry, Invariant 
Differential Operators, and Spherical Functions},
AMS, Providence, RI, 2000 (corrected reprint).

\bibitem{Knapp2002}
A.W.~Knapp,
\emph{Lie Groups Beyond an Introduction},
Progress in Mathematics, Vol.~140, 2nd ed.,
Birkh\"auser, Boston, 2002.

\bibitem{Vladimirov1966}
V.S.~Vladimirov,
\emph{Methods of the Theory of Functions of Many Complex Variables},
MIT Press, Cambridge, MA, 1966.
(Distributional boundary values of holomorphic functions in tube
domains: Chapter~IV, Theorem~25.1.)

\bibitem{NO17}
K.-H.~Neeb and G.~\'Olafsson,
\emph{Antiunitary representations and modular theory},
in: 50th Seminar ``Sophus Lie'', Banach Center Publ., vol.~113 (2017), 291--362.
arXiv:1704.01336.

\bibitem{MN21}
V.~Morinelli and K.-H.~Neeb,
\emph{Covariant homogeneous nets of standard subspaces},
Comm.\ Math.\ Phys.\ \textbf{386} (2021), 305--358.
arXiv:2010.07128.

\bibitem{MN25}
V.~Morinelli and K.-H.~Neeb,
\emph{Nets on coverings of causal symmetric spaces},
in preparation, 2025.

\bibitem{NeebPIM}
K.-H.~Neeb,
\emph{Causal flag manifolds and standard subspaces},
Lecture notes / preprint, Section~5.6, 2024.

\bibitem{NO18}
K.-H.~Neeb and G.~\'Olafsson,
\emph{Reflection positivity: A representation theoretic perspective},
Springer Briefs in Mathematical Physics, vol.~32, Springer, 2018.

\bibitem{BGL02}
R.~Brunetti, D.~Guido, and R.~Longo,
\emph{Modular localization and Wigner particles},
Rev.\ Math.\ Phys.\ \textbf{14} (2002), 759--785.

\end{thebibliography}

\end{document}
